\documentclass[a4paper,11pt]{scrartcl}
\usepackage[a4paper, left=2cm, right=4.5cm, top=2cm, bottom=2cm]{geometry} % kleinere Ränder

%Paket für Header in Koma-Klassen (scrartcl, scrrprt, scrbook, scrlttr2)
\usepackage[headsepline]{scrlayer-scrpage}
% Header groß genug für 3 Zeilen machen
\setlength{\headheight}{3\baselineskip}


% Default Header löschen
\pagestyle{scrheadings}
\clearpairofpagestyles

% nicht kursiv gedruckte header
\setkomafont{pagehead}{\sffamily\upshape}

% Links im Dokuement sowie \url schön machen
\usepackage[colorlinks,pdfpagelabels,pdfstartview = FitH, bookmarksopen = true,bookmarksnumbered = true, linkcolor = black, plainpages = false, hypertexnames = false, citecolor = black]{hyperref}

% Umlaute in der Datei erlauben, auf deutsch umstellen
\usepackage[utf8]{inputenc}
\usepackage[ngerman]{babel}

% Mathesymbole und Ähnliches
\usepackage{amsmath}
\usepackage{mathtools}
\usepackage{amssymb}
\usepackage{microtype}
\newcommand{\NN}{{\mathbb N}}
\newcommand{\RR}{{\mathbb R}}
\newcommand{\QQ}{{\mathbb Q}}
\newcommand{\ZZ}{{\mathbb{Z}}}

% Komplexitätsklassen
\newcommand{\pc}{\ensuremath{{\sf P}}}
\newcommand{\np}{\ensuremath{{\sf NP}}}
\newcommand{\npc}{\ensuremath{{\sf NPC}}}
\newcommand{\pspace}{\ensuremath{{\sf PSPACE}}}
\newcommand{\exptime}{\ensuremath{{\sf EXPTIME}}}
\newcommand{\CClassNP}{\textup{NP}\xspace}
\newcommand{\CClassP}{\textup{P}\xspace}

% Weitere pakete
\usepackage{multicol}
\usepackage{booktabs}

% Abbildungen
\usepackage{tikz}
\usetikzlibrary{arrows,calc}

% Meistens ist \varphi schöner als \phi, genauso bei \theta
\renewcommand{\phi}{\varphi}
\renewcommand{\theta}{\vartheta}

% Aufzählungen anpassen (alternativ: \arabic, \alph)
\renewcommand{\labelenumi}{(\roman{enumi})}

% rwth colors
% colors: blue violet purple carmine red magenta orange yellow grass cyan gold silver
\definecolor{rwth-blue}{cmyk}{1,.5,0,0}\colorlet{rwth-lblue}{rwth-blue!50}\colorlet{rwth-llblue}{rwth-blue!25}
\definecolor{rwth-violet}{cmyk}{.6,.6,0,0}\colorlet{rwth-lviolet}{rwth-violet!50}\colorlet{rwth-llviolet}{rwth-violet!25}
\definecolor{rwth-purple}{cmyk}{.7,1,.35,.15}\colorlet{rwth-lpurple}{rwth-purple!50}\colorlet{rwth-llpurple}{rwth-purple!25}
\definecolor{rwth-carmine}{cmyk}{.25,1,.7,.2}\colorlet{rwth-lcarmine}{rwth-carmine!50}\colorlet{rwth-llcarmine}{rwth-carmine!25}
\definecolor{rwth-red}{cmyk}{.15,1,1,0}\colorlet{rwth-lred}{rwth-red!50}\colorlet{rwth-llred}{rwth-red!25}
\definecolor{rwth-magenta}{cmyk}{0,1,.25,0}\colorlet{rwth-lmagenta}{rwth-magenta!50}\colorlet{rwth-llmagenta}{rwth-magenta!25}
\definecolor{rwth-orange}{cmyk}{0,.4,1,0}\colorlet{rwth-lorange}{rwth-orange!50}\colorlet{rwth-llorange}{rwth-orange!25}
\definecolor{rwth-yellow}{cmyk}{0,0,1,0}\colorlet{rwth-lyellow}{rwth-yellow!50}\colorlet{rwth-llyellow}{rwth-yellow!25}
\definecolor{rwth-grass}{cmyk}{.35,0,1,0}\colorlet{rwth-lgrass}{rwth-grass!50}\colorlet{rwth-llgrass}{rwth-grass!25}
\definecolor{rwth-green}{cmyk}{.7,0,1,0}\colorlet{rwth-lgreen}{rwth-green!50}\colorlet{rwth-llgreen}{rwth-green!25}
\definecolor{rwth-cyan}{cmyk}{1,0,.4,0}\colorlet{rwth-lcyan}{rwth-cyan!50}\colorlet{rwth-llcyan}{rwth-cyan!25}
\definecolor{rwth-teal}{cmyk}{1,.3,.5,.3}\colorlet{rwth-lteal}{rwth-teal!50}\colorlet{rwth-llteal}{rwth-teal!25}
\definecolor{rwth-gold}{cmyk}{.35,.46,.7,.35}
\definecolor{rwth-silver}{cmyk}{.39,.31,.32,.14}

\usepackage{enumitem}

% Header i-> inner (bei einseitig links), c -> center, o -> Outer (bei einseitg rechts)
\ihead{BuK WS 2020/21 \\ Tutorium 04 \\\today}
\chead{\Large Übungsblatt 04}
\ohead{Til Mohr, 405959 \\
	   Andrés Montoya, 405409 \\
	   Marc Ludevid Wulf, 405401}
	
\cfoot*{\pagemark} % Seitenzahlen unten

\begin{document}
	\section*{Aufgabe 4}
	\begin{enumerate}[label=(\alph*)]
	\item	Der Satz von Rice ist hier nicht anwendbar, denn hier geht es darum, \textbf{wie} etwas berechnet wird, und nicht \textbf{was}.\\
			Die Sprache $L_1$ ist rekursiv und kann durch eine TM $M'$ wie folgt entschieden werden:
			\begin{enumerate}[label=\arabic*)]
			\item Berechne $x \coloneqq (2^{\vert \langle M \rangle \vert} - 1) - 1$
			\item Simuliere $M$ für $x$ Schritte
			\item Falls $M$ terminiert, soll $M'$ verwerfen\\
				  Falls $M$ nicht terminiert hat, soll $M'$ akzeptieren
			\end{enumerate}
			Korrektheit:
			\begin{itemize}
			\item Falls $\langle M \rangle \not\in L_1$ $\Rightarrow$ $M$ hält nicht in weniger als $x \coloneqq 2^{\vert \langle M \rangle \vert} - 1$ Schritten $\Rightarrow$ $M'$ verwirft
			\item Falls $\langle M \rangle \not\in L_1$ $\Rightarrow$ $M$ hält in weniger als $x \coloneqq 2^{\vert \langle M \rangle \vert} - 1$ Schritten $\Rightarrow$ $M'$ akzeptiert
			\end{itemize}
	
	\item	Der Satz von Rice ist hier anwendbar, da es um eine partielle Funktion geht.\\
			\begin{align*}
			S &=\{f_M \vert f_M(\langle M \rangle) = 1\}\\
			L_2 &= L(S) = \{\langle M \rangle \vert M \text{ berechnet eine Funktion aus } S\}\\
				&= \{\langle M \rangle \vert M \text{ berechnet auf Eingabe } \langle M \rangle \text{ den Wert } 1 \}
			\end{align*}
			$S \neq \emptyset$, da es in $S$ eine TM $M'$ gibt, die jede Eingabe löscht und dann genau eine $1$ schreibt.\\
			$S \neq R$, da es in $R$ eine TM $M''$ gibt, die jede Eingabe löscht und dann genau eine $0$ schreibt.\\
			Gemäß Satz von Rice ist $L_2$ nicht entscheidbar.
	
	\item	
	
	\end{enumerate}
	
	
	\section*{Aufgabe 5}
	\begin{enumerate}[label=(\alph*)]
	\item	Sei $A$ ein Aufzähler von $L$ mit Ausgabeband ${Ausgabe}_A$. Dann gibt es auch seinen Sparsamen Aufzähler $SA$ mit:
			\begin{itemize}
			\item A
			\item Ausgabeband ${Ausgabe}_{SA}$
			\end{itemize}
			Unser Sparsame Aufzähler geht nun wie folgt vor:
			\begin{enumerate}[label=\arabic*)]
			\item $A$ zählt neues Wort $w$ von $L$ auf seinem Band ${Ausgabe}_A$ auf.
			\item $SA$ ließt nun $w$ und überprüft nun, ob $w$ auf ${Ausgabe}_{SA}$ schon steht.
				\begin{itemize}
				\item[$\rightarrow$] Steht $w$ schon auf ${Ausgabe}_{SA}$, fahre einfach mit $1)$ fort.
				\item[$\rightarrow$] Steht $w$ noch nicht auf ${Ausgabe}_{SA}$, schreibe es dort und fahre mit $1)$ fort.
				\end{itemize}
			\end{enumerate}
			Die Überprüfung, ob ein Wort $w$ bereits auf ${Ausgabe}_{SA}$ steht, geht ja in linearer Zeit.\\
			Damit gibt es für jede rekursiv aufzählbare Sprache $L$ einen sparsamen Aufzähler.
	
	\item	Angenommen die Aussage gilt für rekursiv aufzählbare Sprachen $L$. Also gibt es für $L$ einen kanonisch-organisierten Aufzähler $koA$. Nun kann man mit $koA$ die Sprache $L$ \textbf{entscheiden}, nicht nur erkennen:\\
			Sei $w$ das Wort, welches wir überprüfen wollen:
			\begin{itemize}
			\item Zählt $koA$ das Wort $w$ auf, so ist $w \in L$ (bekannt).
			\item Zählt $koA$ das Wort $w$ noch nicht auf, aber ein Wort $w'$, welches in kanonischer Reihenfolge nach $w$ liegt, so wird $w$ auch nie aufgezählt werden. So ist $w \not\in L$.
			\end{itemize}
			Damit entscheidet $koA$ $L$. Deswegen muss $L$ rekursiv sein. $\Rightarrow$ Widerspruch \\
			Also ist die Aussage falsch.
	\end{enumerate}
	
	
	\section*{Aufgabe 6}
	Das Allgemeine Halteproblem $H_{all} \coloneqq \{\langle M \rangle \vert M \text{ hält auf jeder Eingabe}\}$ ist eine echte Obermenge von $L$, da:
	\begin{itemize}
	\item Alle $\langle M \rangle \in H_{all}$ entscheiden eine rekursive Sprache, da sie ja auf jeder Eingabe halten.
	\item Aus der Vorlesung ist bekannt, dass $H_{all}$ nicht rekursiv aufzählbar ist. Da aber jedes $L$ rekursiv aufzählbar ist, muss es immer eine TM $M'$ geben, mit $\langle M' \rangle \in H_{all}, \langle M' \rangle \not\in L$.
	\end{itemize}
	
\end{document}

