\documentclass[a4paper,11pt]{scrartcl}
\usepackage[a4paper, left=2cm, right=4.5cm, top=2cm, bottom=2cm]{geometry} % kleinere Ränder

%Paket für Header in Koma-Klassen (scrartcl, scrrprt, scrbook, scrlttr2)
\usepackage[headsepline]{scrlayer-scrpage}
% Header groß genug für 3 Zeilen machen
\setlength{\headheight}{3\baselineskip}


% Default Header löschen
\pagestyle{scrheadings}
\clearpairofpagestyles

% nicht kursiv gedruckte header
\setkomafont{pagehead}{\sffamily\upshape}

% Links im Dokuement sowie \url schön machen
\usepackage[colorlinks,pdfpagelabels,pdfstartview = FitH, bookmarksopen = true,bookmarksnumbered = true, linkcolor = black, plainpages = false, hypertexnames = false, citecolor = black]{hyperref}

% Umlaute in der Datei erlauben, auf deutsch umstellen
\usepackage[utf8]{inputenc}
\usepackage[ngerman]{babel}

% Mathesymbole und Ähnliches
\usepackage{amsmath}
\usepackage{mathtools}
\usepackage{amssymb}
\usepackage{microtype}
\newcommand{\NN}{{\mathbb N}}
\newcommand{\RR}{{\mathbb R}}
\newcommand{\QQ}{{\mathbb Q}}
\newcommand{\ZZ}{{\mathbb{Z}}}

% Komplexitätsklassen
\newcommand{\pc}{\ensuremath{{\sf P}}}
\newcommand{\np}{\ensuremath{{\sf NP}}}
\newcommand{\npc}{\ensuremath{{\sf NPC}}}
\newcommand{\pspace}{\ensuremath{{\sf PSPACE}}}
\newcommand{\exptime}{\ensuremath{{\sf EXPTIME}}}
\newcommand{\CClassNP}{\textup{NP}\xspace}
\newcommand{\CClassP}{\textup{P}\xspace}

% Weitere pakete
\usepackage{multicol}
\usepackage{booktabs}

% Abbildungen
\usepackage{tikz}
\usetikzlibrary{arrows,calc}

% Meistens ist \varphi schöner als \phi, genauso bei \theta
\renewcommand{\phi}{\varphi}
\renewcommand{\theta}{\vartheta}

% Aufzählungen anpassen (alternativ: \arabic, \alph)
\renewcommand{\labelenumi}{(\roman{enumi})}

% rwth colors
% colors: blue violet purple carmine red magenta orange yellow grass cyan gold silver
\definecolor{rwth-blue}{cmyk}{1,.5,0,0}\colorlet{rwth-lblue}{rwth-blue!50}\colorlet{rwth-llblue}{rwth-blue!25}
\definecolor{rwth-violet}{cmyk}{.6,.6,0,0}\colorlet{rwth-lviolet}{rwth-violet!50}\colorlet{rwth-llviolet}{rwth-violet!25}
\definecolor{rwth-purple}{cmyk}{.7,1,.35,.15}\colorlet{rwth-lpurple}{rwth-purple!50}\colorlet{rwth-llpurple}{rwth-purple!25}
\definecolor{rwth-carmine}{cmyk}{.25,1,.7,.2}\colorlet{rwth-lcarmine}{rwth-carmine!50}\colorlet{rwth-llcarmine}{rwth-carmine!25}
\definecolor{rwth-red}{cmyk}{.15,1,1,0}\colorlet{rwth-lred}{rwth-red!50}\colorlet{rwth-llred}{rwth-red!25}
\definecolor{rwth-magenta}{cmyk}{0,1,.25,0}\colorlet{rwth-lmagenta}{rwth-magenta!50}\colorlet{rwth-llmagenta}{rwth-magenta!25}
\definecolor{rwth-orange}{cmyk}{0,.4,1,0}\colorlet{rwth-lorange}{rwth-orange!50}\colorlet{rwth-llorange}{rwth-orange!25}
\definecolor{rwth-yellow}{cmyk}{0,0,1,0}\colorlet{rwth-lyellow}{rwth-yellow!50}\colorlet{rwth-llyellow}{rwth-yellow!25}
\definecolor{rwth-grass}{cmyk}{.35,0,1,0}\colorlet{rwth-lgrass}{rwth-grass!50}\colorlet{rwth-llgrass}{rwth-grass!25}
\definecolor{rwth-green}{cmyk}{.7,0,1,0}\colorlet{rwth-lgreen}{rwth-green!50}\colorlet{rwth-llgreen}{rwth-green!25}
\definecolor{rwth-cyan}{cmyk}{1,0,.4,0}\colorlet{rwth-lcyan}{rwth-cyan!50}\colorlet{rwth-llcyan}{rwth-cyan!25}
\definecolor{rwth-teal}{cmyk}{1,.3,.5,.3}\colorlet{rwth-lteal}{rwth-teal!50}\colorlet{rwth-llteal}{rwth-teal!25}
\definecolor{rwth-gold}{cmyk}{.35,.46,.7,.35}
\definecolor{rwth-silver}{cmyk}{.39,.31,.32,.14}

\usepackage{enumitem}

\usepackage{listings}  
\lstset{columns=fullflexible,
        mathescape=true,
        morekeywords={if,else,return,while,Input}
        }

% Header i-> inner (bei einseitig links), c -> center, o -> Outer (bei einseitg rechts)
\ihead{BuK WS 2020/21 \\ Tutorium 04 \\\today}
\chead{\Large Weihnachtsblatt}
\ohead{Til Mohr, 405959 \\
	   Andrés Montoya, 405409 \\
	   Marc Ludevid Wulf, 405401}
	
\cfoot*{\pagemark} % Seitenzahlen unten

\begin{document}
	\section*{Aufgabe 1}
	Damit $L$ entscheidbar ist, muss es eine TM $M$ geben, sodass $L(M) = L$. $M$ würde also entscheiden, ob der Weihnachtsmann existiert oder nicht.\\
	$L$ ist damit genau dann entscheidbar, wenn wir wissen, ob der Weihnachtsmann existiert oder nicht.
	
	
	\section*{Aufgabe 2}
	\subsection*{Formalisierung}
	Wir haben Eingabewörter $w_i \in \{0,1\}^*$ und TMs $M_i$.
	Eine Eingabe $w_i$ wird genau dann akzeptiert, wenn alle $M_j$ $w_i$ akzeptieren mit $0 \leq j \leq i$.
	\textit{Kann man entscheiden, welche $w_i$ alle akzeptiert werden?}\\
	$\Rightarrow$ Diagonalisierung
	
	\subsection*{Lösung}
	Konstruiere Mehrband-Band-TM $M'$ mit folgender Funktionsweise auf einer Eingabe $w_i$ auf Band 0:
	\begin{enumerate}
	\item	Entnehme $i$ aus der Eingabe $w_i$ und hinterlege es auf einem Band als $j$.
	\item	Lies $j$ und bekomme $M_j$.
	\item	Führe auf der Originaleingabe $w_j$ auf $M_ij$ aus.
		\begin{itemize}
		\item	Falls $M_j$ verwirft, soll auch $M'$ verwerfen.
		\item	Falls $M_j$ akzeptiert und $j=0$, soll $M'$ verwerfen.
		\item	Falls $M_j$ akzeptiert und $j>0$, soll $j$ um $1$ reduziert werden, dann ab \verb|2)| fortgefahren werden.
		\end{itemize}
	\end{enumerate}
	
	\subsubsection*{Korrektheit}
	$w_i$ wird akzeptiert $\Rightarrow$ Alle $M_j$ mit $0 \leq j \leq i$ akzeptieren $w_i$ $\Rightarrow$ $M'$ akzeptiert $w_i$.\\
	$w_i$ wird verworfen $\Rightarrow$ Es mind. $j$ in $0 \leq j \leq i$, sodass $M_j$ auf $w_i$ verwirft $\Rightarrow$ $M'$ verwirft $w_i$.
	
	\section*{Aufgabe 3}
	\begin{enumerate}[label=\alph*)]
	\item	Ja, $L_1$ ist entscheidbar. Denn sei $M'$ eine TM. $M'$ kann sich einfach die Codierung von jedem $\langle M \rangle$ ansehen, bei unter 24 Zuständen akzeptieren, sonst verwerfen. Damit erkennt $M'$ $L_1$.
	
	\item	Nein. Halteproblem $\leq$ $L_2$
	
	\item	
	\end{enumerate}
	
	\section*{Aufgabe 4}
	
\end{document}

