\documentclass[a4paper,11pt]{scrartcl}
\usepackage[a4paper, left=2cm, right=4.5cm, top=2cm, bottom=2cm]{geometry} % kleinere Ränder

%Paket für Header in Koma-Klassen (scrartcl, scrrprt, scrbook, scrlttr2)
\usepackage[headsepline]{scrlayer-scrpage}
% Header groß genug für 3 Zeilen machen
\setlength{\headheight}{3\baselineskip}


% Default Header löschen
\pagestyle{scrheadings}
\clearpairofpagestyles

% nicht kursiv gedruckte header
\setkomafont{pagehead}{\sffamily\upshape}

% Links im Dokuement sowie \url schön machen
\usepackage[colorlinks,pdfpagelabels,pdfstartview = FitH, bookmarksopen = true,bookmarksnumbered = true, linkcolor = black, plainpages = false, hypertexnames = false, citecolor = black]{hyperref}

% Umlaute in der Datei erlauben, auf deutsch umstellen
\usepackage[utf8]{inputenc}
\usepackage[ngerman]{babel}

% Mathesymbole und Ähnliches
\usepackage{amsmath}
\usepackage{mathtools}
\usepackage{amssymb}
\usepackage{microtype}
\newcommand{\NN}{{\mathbb N}}
\newcommand{\RR}{{\mathbb R}}
\newcommand{\QQ}{{\mathbb Q}}
\newcommand{\ZZ}{{\mathbb{Z}}}

% Komplexitätsklassen
\newcommand{\pc}{\ensuremath{{\sf P}}}
\newcommand{\np}{\ensuremath{{\sf NP}}}
\newcommand{\npc}{\ensuremath{{\sf NPC}}}
\newcommand{\pspace}{\ensuremath{{\sf PSPACE}}}
\newcommand{\exptime}{\ensuremath{{\sf EXPTIME}}}
\newcommand{\CClassNP}{\textup{NP}\xspace}
\newcommand{\CClassP}{\textup{P}\xspace}

% Weitere pakete
\usepackage{multicol}
\usepackage{booktabs}

% Abbildungen
\usepackage{tikz}
\usetikzlibrary{arrows,calc}

% Meistens ist \varphi schöner als \phi, genauso bei \theta
\renewcommand{\phi}{\varphi}
\renewcommand{\theta}{\vartheta}

% Aufzählungen anpassen (alternativ: \arabic, \alph)
\renewcommand{\labelenumi}{(\roman{enumi})}

% rwth colors
% colors: blue violet purple carmine red magenta orange yellow grass cyan gold silver
\definecolor{rwth-blue}{cmyk}{1,.5,0,0}\colorlet{rwth-lblue}{rwth-blue!50}\colorlet{rwth-llblue}{rwth-blue!25}
\definecolor{rwth-violet}{cmyk}{.6,.6,0,0}\colorlet{rwth-lviolet}{rwth-violet!50}\colorlet{rwth-llviolet}{rwth-violet!25}
\definecolor{rwth-purple}{cmyk}{.7,1,.35,.15}\colorlet{rwth-lpurple}{rwth-purple!50}\colorlet{rwth-llpurple}{rwth-purple!25}
\definecolor{rwth-carmine}{cmyk}{.25,1,.7,.2}\colorlet{rwth-lcarmine}{rwth-carmine!50}\colorlet{rwth-llcarmine}{rwth-carmine!25}
\definecolor{rwth-red}{cmyk}{.15,1,1,0}\colorlet{rwth-lred}{rwth-red!50}\colorlet{rwth-llred}{rwth-red!25}
\definecolor{rwth-magenta}{cmyk}{0,1,.25,0}\colorlet{rwth-lmagenta}{rwth-magenta!50}\colorlet{rwth-llmagenta}{rwth-magenta!25}
\definecolor{rwth-orange}{cmyk}{0,.4,1,0}\colorlet{rwth-lorange}{rwth-orange!50}\colorlet{rwth-llorange}{rwth-orange!25}
\definecolor{rwth-yellow}{cmyk}{0,0,1,0}\colorlet{rwth-lyellow}{rwth-yellow!50}\colorlet{rwth-llyellow}{rwth-yellow!25}
\definecolor{rwth-grass}{cmyk}{.35,0,1,0}\colorlet{rwth-lgrass}{rwth-grass!50}\colorlet{rwth-llgrass}{rwth-grass!25}
\definecolor{rwth-green}{cmyk}{.7,0,1,0}\colorlet{rwth-lgreen}{rwth-green!50}\colorlet{rwth-llgreen}{rwth-green!25}
\definecolor{rwth-cyan}{cmyk}{1,0,.4,0}\colorlet{rwth-lcyan}{rwth-cyan!50}\colorlet{rwth-llcyan}{rwth-cyan!25}
\definecolor{rwth-teal}{cmyk}{1,.3,.5,.3}\colorlet{rwth-lteal}{rwth-teal!50}\colorlet{rwth-llteal}{rwth-teal!25}
\definecolor{rwth-gold}{cmyk}{.35,.46,.7,.35}
\definecolor{rwth-silver}{cmyk}{.39,.31,.32,.14}

\usepackage{enumitem}

\usepackage{listings}  
\lstset{columns=fullflexible,
        mathescape=true,
        morekeywords={if,else,return,while,Input}
        }

% Header i-> inner (bei einseitig links), c -> center, o -> Outer (bei einseitg rechts)
\ihead{BuK WS 2020/21 \\ Tutorium 04 \\\today}
\chead{\Large Übungsblatt 04}
\ohead{Til Mohr, 405959 \\
	   Andrés Montoya, 405409 \\
	   Marc Ludevid Wulf, 405401}
	
\cfoot*{\pagemark} % Seitenzahlen unten

\begin{document}
	\section*{Aufgabe 4}
	\begin{enumerate}[label=\arabic*)]
	\item	$\left[ \frac{ab}{abb} \right] \left[ \frac{bb}{b} \right]$ ist eine mögliche Lösung dieser PKP-Instanz.
	
	\item	\begin{itemize}
			\item $\left[ \frac{a}{ba} \right], \left[ \frac{a}{bb} \right], \left[ \frac{aab}{ab} \right], \left[ \frac{abab}{aa} \right]$ sind keine möglichen Startdominos, da sie mit unterschiedlichen Buchstaben beginnen.
			\item $\left[ \frac{ab}{abb} \right], \left[ \frac{aa}{aab} \right]$ erzeugen zwar selber gleich beginnende Worte oben und unten, jedoch endet bei beiden jedes Wort unten mit einem $b$ mehr als oben. Zudem gibt es keinen Dominostein, der oben mit einem $b$ beginnt. Daher sind diese auch keine möglichen Startdominos.
			\end{itemize}
			Da es keine möglichen Startdominos gibt, hat diese PKP-Instanz keine Lösung.	
	\end{enumerate}
		
	\section*{Aufgabe 5}
	Um die Aussage zu beweisen, zeigen wir erst, dass $L_{01}$ rekursiv ist:\\
	Sei $T$ eine TM mit folgender Funktionsweise:
	\begin{enumerate}
	\item Gehe an den Anfang des Eingabewortes.
	\item Ist der Buchstabe unter dem Kopf eine $1$, so verwirf die Eingabe. Ist der Buchstabe unter dem Kopf ein $B$, so verwirf die Eingabe (Eingabe war $\epsilon$). Ist der Buchstabe unter dem Kopf eine $0$, so lösche das Zeichen, und gehe zu dem Ende des Eingabewortes.
	\item Ist der Buchstabe unter dem Kopf eine $0$, so verwirf die Eingabe. Ist der Buchstabe unter dem Kopf eine $1$, so lösche das Zeichen und gehe einen Schritt nach Links.
	\item Ist das Zeichen unter dem Kopf $B$, akzeptiere die Eingabe. Sonst fahre bei Schritt \textit{(i)} fort.
	\end{enumerate}
	Korrektheit:
	\begin{itemize}
	\item Sei $\epsilon$ das Eingabewort, $\epsilon \not\in L_{01}$ $\Rightarrow$ $T$ verwirft die Eingabe sofort.
	\item Sei $w$ das Eingabewort, $w \not\in L_{01}$ $\Rightarrow$ Durch das gleichmäßige Abbauen des Eingabewortes an beiden Seiten erkennt $T$ in Schritten \textit{(ii)} bzw. \textit{(iii)} irgendwann, dass die Eingabe nicht dem Format $0^n1^n, n>0$ entspricht $\Rightarrow$ $T$ verwirft
	\item Sei $w$ das Eingabewort, $w \in L_{01}$ $\Rightarrow$ Duch das gleichmäßige Abbauen des Eingabewortes an beiden Seiten wird $T$ keine Fehler des Formates $0^n1^n, n>0$ an $w$ entdecken $\Rightarrow$ $T$ akzeptiert
	\end{itemize}
	$T$ erkennt offensichtlich $L_{01}$, daher ist $L_{01}$ rekursiv.	
	
	Sei nun $L$ eine Sprache.
	\begin{itemize}
	\item[$\Rightarrow$] Sei $L$ rekursiv. Dann gilt auch $L \leq L_{01}$, da ja auch $L_{01}$ rekursiv ist. (VL)
	\item[$\Leftarrow$] Gelte $L \leq L_{01}$. Da ja $L_{01}$ rekursiv ist, muss auch $L$ rekursiv sein. (VL)
	\end{itemize}
	Damit gilt die Aussage.
	
\newpage
	\section*{Aufgabe 6}
	\begin{enumerate}[label=\alph*)]
	\item	
			
	\item	Dieses Problem ist gleich dem PKP.\\
			Sei $K$ eine PKP-Instanz, wobei hier $k_i$ der $i$-ten Stein aus $K$ ist.\\
			Aus $K$ können wir nun eine Instanz $K'$ unserer PKP-Variante konstruieren. Dazu wende folgendes Verfahren an:
			\begin{itemize}
			\item Hat der $k_i$ Stein oben und unten verschieden lange Wörter, so übernehme diesen auch in $K'$.
			\item Hat der $k_i$ Stein gleich lange Wörter oben und unten, dann ist der Stein der Form $\left[ \frac{a_1...a_n}{b_1...b_n} \right]$ mit $a_j, b_j \in \Sigma$ für $0<j\leq n$.\\
				  Nehme für den Stein $k_i$ einen Buchstaben $\#_i \not\in\Sigma$ und baue aus $k_i$ 2 neue Steine und füge diese dann $K'$ hinzu:\\
				  $x_i\coloneqq \left[ \frac{\#_ia_1...a_n}{\#_i} \right], y_i\coloneqq \left[ \frac{\#_i}{b_1...b_n\#_i} \right]$\\
				  Damit muss immer auf ein $x_i$ ein $y_i$ folgen
			\end{itemize}
			Gibt es also eine Folge $\langle o_1,...,o_m \rangle$, sodass $k_{o_1} ... k_{o_m}$ eine Lösung von $K$ ist, dann gibt es eine Folge $\langle o'_1,...,o'_p \rangle , p \geq m$, sodass $k'_{o_1} ... k'_{o_m}$ eine Lösung von $K'$ ist:
			\begin{lstlisting}
			Input: $\langle o_1,...,o_m \rangle$
			i = 1
			I = $\langle \rangle$
			while(i < m)
				if($k_{o_i}$ oben und unten verschieden lang)
					I += index($k_{o_i}$)
				else
					I += index($x_{o_i}$)
					I += index($y_{o_i}$)
				i++
			return I
			\end{lstlisting}
			Also ist diese Problem auch nicht entscheidbar.
	\end{enumerate}
	
\end{document}

