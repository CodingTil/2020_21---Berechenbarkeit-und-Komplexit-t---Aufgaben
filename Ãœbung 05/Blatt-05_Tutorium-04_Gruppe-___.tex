\documentclass[a4paper,11pt]{scrartcl}
\usepackage[a4paper, left=2cm, right=4.5cm, top=2cm, bottom=2cm]{geometry} % kleinere Ränder

%Paket für Header in Koma-Klassen (scrartcl, scrrprt, scrbook, scrlttr2)
\usepackage[headsepline]{scrlayer-scrpage}
% Header groß genug für 3 Zeilen machen
\setlength{\headheight}{3\baselineskip}


% Default Header löschen
\pagestyle{scrheadings}
\clearpairofpagestyles

% nicht kursiv gedruckte header
\setkomafont{pagehead}{\sffamily\upshape}

% Links im Dokuement sowie \url schön machen
\usepackage[colorlinks,pdfpagelabels,pdfstartview = FitH, bookmarksopen = true,bookmarksnumbered = true, linkcolor = black, plainpages = false, hypertexnames = false, citecolor = black]{hyperref}

% Umlaute in der Datei erlauben, auf deutsch umstellen
\usepackage[utf8]{inputenc}
\usepackage[ngerman]{babel}

% Mathesymbole und Ähnliches
\usepackage{amsmath}
\usepackage{mathtools}
\usepackage{amssymb}
\usepackage{microtype}
\newcommand{\NN}{{\mathbb N}}
\newcommand{\RR}{{\mathbb R}}
\newcommand{\QQ}{{\mathbb Q}}
\newcommand{\ZZ}{{\mathbb{Z}}}

% Komplexitätsklassen
\newcommand{\pc}{\ensuremath{{\sf P}}}
\newcommand{\np}{\ensuremath{{\sf NP}}}
\newcommand{\npc}{\ensuremath{{\sf NPC}}}
\newcommand{\pspace}{\ensuremath{{\sf PSPACE}}}
\newcommand{\exptime}{\ensuremath{{\sf EXPTIME}}}
\newcommand{\CClassNP}{\textup{NP}\xspace}
\newcommand{\CClassP}{\textup{P}\xspace}

% Weitere pakete
\usepackage{multicol}
\usepackage{booktabs}

% Abbildungen
\usepackage{tikz}
\usetikzlibrary{arrows,calc}

% Meistens ist \varphi schöner als \phi, genauso bei \theta
\renewcommand{\phi}{\varphi}
\renewcommand{\theta}{\vartheta}

% Aufzählungen anpassen (alternativ: \arabic, \alph)
\renewcommand{\labelenumi}{(\roman{enumi})}

% rwth colors
% colors: blue violet purple carmine red magenta orange yellow grass cyan gold silver
\definecolor{rwth-blue}{cmyk}{1,.5,0,0}\colorlet{rwth-lblue}{rwth-blue!50}\colorlet{rwth-llblue}{rwth-blue!25}
\definecolor{rwth-violet}{cmyk}{.6,.6,0,0}\colorlet{rwth-lviolet}{rwth-violet!50}\colorlet{rwth-llviolet}{rwth-violet!25}
\definecolor{rwth-purple}{cmyk}{.7,1,.35,.15}\colorlet{rwth-lpurple}{rwth-purple!50}\colorlet{rwth-llpurple}{rwth-purple!25}
\definecolor{rwth-carmine}{cmyk}{.25,1,.7,.2}\colorlet{rwth-lcarmine}{rwth-carmine!50}\colorlet{rwth-llcarmine}{rwth-carmine!25}
\definecolor{rwth-red}{cmyk}{.15,1,1,0}\colorlet{rwth-lred}{rwth-red!50}\colorlet{rwth-llred}{rwth-red!25}
\definecolor{rwth-magenta}{cmyk}{0,1,.25,0}\colorlet{rwth-lmagenta}{rwth-magenta!50}\colorlet{rwth-llmagenta}{rwth-magenta!25}
\definecolor{rwth-orange}{cmyk}{0,.4,1,0}\colorlet{rwth-lorange}{rwth-orange!50}\colorlet{rwth-llorange}{rwth-orange!25}
\definecolor{rwth-yellow}{cmyk}{0,0,1,0}\colorlet{rwth-lyellow}{rwth-yellow!50}\colorlet{rwth-llyellow}{rwth-yellow!25}
\definecolor{rwth-grass}{cmyk}{.35,0,1,0}\colorlet{rwth-lgrass}{rwth-grass!50}\colorlet{rwth-llgrass}{rwth-grass!25}
\definecolor{rwth-green}{cmyk}{.7,0,1,0}\colorlet{rwth-lgreen}{rwth-green!50}\colorlet{rwth-llgreen}{rwth-green!25}
\definecolor{rwth-cyan}{cmyk}{1,0,.4,0}\colorlet{rwth-lcyan}{rwth-cyan!50}\colorlet{rwth-llcyan}{rwth-cyan!25}
\definecolor{rwth-teal}{cmyk}{1,.3,.5,.3}\colorlet{rwth-lteal}{rwth-teal!50}\colorlet{rwth-llteal}{rwth-teal!25}
\definecolor{rwth-gold}{cmyk}{.35,.46,.7,.35}
\definecolor{rwth-silver}{cmyk}{.39,.31,.32,.14}

\usepackage{enumitem}

% Header i-> inner (bei einseitig links), c -> center, o -> Outer (bei einseitg rechts)
\ihead{BuK WS 2020/21 \\ Tutorium 04 \\\today}
\chead{\Large Übungsblatt 04}
\ohead{Til Mohr, 405959 \\
	   Andrés Montoya, 405409 \\
	   Marc Ludevid Wulf, 405401}
	
\cfoot*{\pagemark} % Seitenzahlen unten

\begin{document}
	\section*{Aufgabe 4}
	Da $A \leq B$ gilt und $B$ rekursiv aufzählbar ist, ist auch $A$ rekursiv aufzählbar.\\
	Damit also $A$ entscheidbar ist, muss noch $\bar{A}$ rekursiv aufzählbar sein.\\
	Jedoch kann $B \leq \bar{A}$ auch gelten, wenn $\bar{A}$ nicht rekursiv aufzählbar ist (beispielsweise wenn $B$ nicht rekursiv ist):
	\begin{align*}
	B \text{ nicht rekursiv } 	&\Rightarrow \bar{A} \text{ nicht rekursiv} \\
								&\Rightarrow \bar{A} \text{ nicht rekursiv aufzählbar (da A rekursiv aufzählbar)}
	\end{align*}
	Damit ist auch $A$ nicht immer rekursiv.
	
		
	\section*{Aufgabe 5}
	Zu zeigen: für eine berechenbare Funktion $f$ gilt:\\
	$x \in H_{\epsilon} \Leftrightarrow f(x) \in L_{111}$ und $x \not \in H_{\epsilon} \Leftrightarrow f(x) \not \in L_{111}$\\
	Sei f:\\
	$f(x) = \langle M' \rangle$ falls $x$ Gödelnummer $\langle M \rangle$ wobei $M'$ die TM ist die zunächst das ganze Band löscht und dann $M$ ausführt.\\
	$f(x) = \langle M'' \rangle$ falls $x$ keine Gödelnummer ist, wobei $M''$ eine TM ist die nur einen Endlosschleife hat und somit nie terminiert.\\
	$f$ ist offensichtlich berechenbar, da das Bestimmen ob ein Eingabewort eine TM ist, das Löschen des Bandes und das Simulieren einer TM alles berechenbar ist.\\

	$x \in H_{\epsilon}$\\
			   $\Leftrightarrow x$ ist Gödelnummer $\langle M \rangle$ und $\langle M \rangle$ hält auf $\epsilon$\\
			   $\Leftrightarrow \langle M' \rangle$ hält auf jeder Eingabe somit auch die, die auf 111 enden.\\
			   $\Leftrightarrow \langle M' \rangle \in L_{111}$\\
			   $\Leftrightarrow f(x) \in L_{111}$\\

	$x \not \in H_{\epsilon}$\\
				 $\Leftrightarrow x$ ist keine Gödelnummer oder $x$ ist Gödelnummer $\langle M \rangle$ und $\langle M \rangle$ hält nicht auf $\epsilon$\\
				 $\Leftrightarrow f(x) = \langle M'' \rangle$ oder $f(x) = \langle M' \rangle$ wobei $M'$ auf keiner Eingabe terminiert, somit auch die, die auf 111 enden.\\
				 $\Leftrightarrow f(x)$ terminiert nie.\\
				 $\Leftrightarrow f(x) \not \in L_{111}$\\

	Somit gilt: $H_{\epsilon} \leq L_{111}$
	
	\section*{Aufgabe 6}
	\begin{enumerate}[label=\alph*)]
	\item	Satz von Rice:
			\begin{align*}
			L_{\mathbb{P}} &= \{\langle M \rangle \vert L(M) = \mathbb{P}\} \\
			S &= \{f_M \vert \forall_{p\in\mathbb{P}}: f_M(p) = 1 \land \forall_{q\not\in\mathbb{P}}: f_M(q) = 0\}		
			\end{align*}
			$S \neq R$, da in $S$ nicht die Funktion enthalten ist, die für alle Eingaben $0$ ausgibt.\\
			$S \neq \emptyset$, da es Algorithmen gibt um zu bestimmen, ob eine bestimmte Zahl $x$ eine Primzahl ist.\\
			Ausserdem macht die Sprache ausschliesslich eine Aussage über die Ausgabe der TM.\\
			Nach dem Satz von Rice ist also $L_{\mathbb{P}} = L(S)$ nicht rekursiv.
\newpage
	\item	Sei $L_1$ das Entscheidungsproblem: Gegeben eine TM $M$, akzeptiert diese jede Eingabe $w$? Also:
			\begin{center}
			$L_1 = \{\langle M \rangle \vert L(M) = \Sigma^*\}$\\
			\end{center}
			Mit dem Satz von Rice ist schnell bewiesen, dass dieses Entscheidungsproblem unentscheidbar ist:\\
			$S \neq R$, da in $S$ nicht die Funktion enthalten ist, die für alle Eingaben $0$ ausgibt.\\
			$S \neq \emptyset$, da in $S$ die Funktion enthalten ist, die fül alle Eingabe $1$ ausgibt.\\
			Da die Sprache ausschliesslich eine Aussage über die Ausgabe der TM macht, $S \neq R$ und $S \neq \emptyset$ gilt, besagt der Satz von Rice dass $L_1$ nicht entscheidbar ist.\\

			Nun zeigen wir, dass wenn es eine TM $M_{comp}$ gäbe die die gegebene Sprache $L_{comp}$ entscheidet man dann durch Unterprogrammtechnik auch $L_1$ mit der TM $M_1$ entscheiden könnte.
			Da wir gerade gezeigt haben, dass $L_1$ nicht entscheidbar ist kann $L_{comp}$ also ebenfalls nicht entscheidbar sein.
			
			Dafür definieren wir die TM $M_0$ welche jedes Wort verwirft.\\
			Die TM $M_1$ gibt einfach $\langle M_0 \rangle$ und das Eingabewort in die TM $M_{comp}$ ein und übernimmt dann die Ausgabe.\\
			
			Korrektheit:\\
			\\
			$\overline{L(M_0)} = \{\langle M \rangle \vert L(M) = \Sigma^*\} = L_1$\\
			\\
			$x \in L_1$\\
			$\Leftrightarrow x$ ist Gödelnummer $\langle M \rangle$ und $L(M) = L_1$\\
			$\Leftrightarrow M_{comp}$ akzeptiert mit Eingaben $\langle M \rangle$ und $\langle M_0 \rangle$\\
			$\Leftrightarrow M_1$ akzeptiert ebenfalls.\\
			$\Leftrightarrow x \in L(M_1)$\\
			\\
			$x \not\in L_1$\\
			$\Leftrightarrow x$ ist keine Gödelnummer oder $x$ ist Gödelnummer $\langle M \rangle$ und $L(M) \neq L_1$\\
			$\Leftrightarrow M_{comp}$ verwirft die Eingaben $\langle M \rangle$ und $\langle M_0 \rangle$\\
			$\Leftrightarrow M_1$ verwirft ebenfalls.\\
			$\Leftrightarrow x \not\in L(M_1)$\\
			
	\end{enumerate}
		
	\newpage
	
	\section*{Aufgabe 7}
	rekursiv aufzählbar = ra, rekursiv = r
	\begin{enumerate}[label=\alph*)]
	\item	\begin{itemize}
			\item[$"\Rightarrow"$]
				$L$ ra $\Rightarrow$ Es gibt eine TM $M$, die für alle $x\in L$ hält und akzeptiert.\\
				$\Rightarrow$ Konstruiere TM $M'$, die $M$ simuliert:
					\begin{itemize}
					\item $M$ akzeptiert $\Rightarrow$ $M'$ akzeptiert
					\item $M$ hält nicht $\Rightarrow$ $M'$ hält nicht
					\item $M$ verwirft $\Rightarrow$ Setze $M'$ in eine Endlosschleife $\Rightarrow$ $M'$ hält nicht
					\end{itemize}
				$\Rightarrow$ $L(M)=L(M')$\\
				$\Rightarrow$ Es gibt also eine partielle berechenbare Funktion f, die von $M'$ (bzw. gleich $M$) berechnet wird\\
				$\Rightarrow$ $\operatorname{Def}(f)=\{x \vert f(x) \neq \perp\}=L(M')=L(M)=L$
			
			\item[$"\Leftarrow"$]
				$f$ eine partielle berechenbare Funktion mit $L \coloneqq \operatorname{Def}(f)=\{x \vert f(x) \neq \perp\}$\\
				$\Rightarrow$ Es gibt eine TM $M$, die $f$ berechnet
				$\Rightarrow$ $L(M)=L$, da:
					\begin{itemize}
					\item $M$ hält auf x $\Leftrightarrow$ $f(x) \neq \perp$ $\Leftrightarrow$ $x\in L$
					\item $M$ hält nicht auf x $\Leftrightarrow$ $f(x) = \perp$ $\Leftrightarrow$ $x\not\in L$
					\end{itemize}
				Also ist $L$ ra.
			\end{itemize}
			Damit ist die Aussage folglich bewiesen.
			
	\item	\begin{itemize}
			\item[$"\Rightarrow"$]
				$L$ rekursiv aufzählbar:
				\begin{enumerate}[label=\arabic*)]
				\item	Falls  $L = \emptyset$ gilt (immer ra)
				\item	Falls $L \neq \emptyset$:\\
						Es gibt ja $h: \Sigma^* \rightarrow \mathbb{N}$.\\
						Zudem hat die ra Sprache $L$ einen Aufzähler $A$, der Wörter $w \in L$ aufzählt.\\
						$\Rightarrow$ Es gibt $g: \mathbb{N} \rightarrow L$, die eine Zuordnung von Zahlen zu den Wörter auf dem Ausgabeband von $A$ ist
						$\Rightarrow$ $f \coloneqq g \circ h$ ist somit eine totale Funktion von $\Sigma^*$ nach $L$
				\end{enumerate}
			
			\item[$"\Leftarrow"$]
				\begin{enumerate}[label=\arabic*)]
				\item	$L=\emptyset$ $\Rightarrow$ $L$ r $\Rightarrow$ $L$ ra (offensichtlich)
				\item	$f: \Sigma^* \rightarrow L$ solche totale Funktion\\
						$\Rightarrow$ Sei $T$ eine TM, die $f$ berechnet für $w \in \Sigma^*$\\
						$\Rightarrow$ Sei $Z$ ein Aufzähler von $\Sigma^*$\\
						$\Rightarrow$ Sei $T'$ eine TM, die $T$ und $Z$ benutzt, keine Eingaben nimmt, jedes von $Z$ geschriebene Wort in $T$ füttert.\\
						$\Rightarrow$ $T'$ ist ein Aufzähler von $L$\\
						$\Rightarrow$ $L$ ist ra
				\end{enumerate}
			
			\end{itemize}
	\end{enumerate}
	
	
\end{document}

