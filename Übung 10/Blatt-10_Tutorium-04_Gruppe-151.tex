\documentclass[a4paper,11pt]{scrartcl}
\usepackage[a4paper, left=2cm, right=4.5cm, top=2cm, bottom=2cm]{geometry} % kleinere Ränder

%Paket für Header in Koma-Klassen (scrartcl, scrrprt, scrbook, scrlttr2)
\usepackage[headsepline]{scrlayer-scrpage}
% Header groß genug für 3 Zeilen machen
\setlength{\headheight}{3\baselineskip}


% Default Header löschen
\pagestyle{scrheadings}
\clearpairofpagestyles

% nicht kursiv gedruckte header
\setkomafont{pagehead}{\sffamily\upshape}

% Links im Dokuement sowie \url schön machen
\usepackage[colorlinks,pdfpagelabels,pdfstartview = FitH, bookmarksopen = true,bookmarksnumbered = true, linkcolor = black, plainpages = false, hypertexnames = false, citecolor = black]{hyperref}

% Umlaute in der Datei erlauben, auf deutsch umstellen
\usepackage[utf8]{inputenc}
\usepackage[ngerman]{babel}

% Mathesymbole und Ähnliches
\usepackage{amsmath}
\usepackage{mathtools}
\usepackage{amssymb}
\usepackage{microtype}
\newcommand{\NN}{{\mathbb N}}
\newcommand{\RR}{{\mathbb R}}
\newcommand{\QQ}{{\mathbb Q}}
\newcommand{\ZZ}{{\mathbb{Z}}}

% Komplexitätsklassen
\newcommand{\pc}{\ensuremath{{\sf P}}}
\newcommand{\np}{\ensuremath{{\sf NP}}}
\newcommand{\npc}{\ensuremath{{\sf NPC}}}
\newcommand{\pspace}{\ensuremath{{\sf PSPACE}}}
\newcommand{\exptime}{\ensuremath{{\sf EXPTIME}}}
\newcommand{\CClassNP}{\textup{NP}\xspace}
\newcommand{\CClassP}{\textup{P}\xspace}

% Weitere pakete
\usepackage{multicol}
\usepackage{booktabs}

% Abbildungen
\usepackage{tikz}
\usetikzlibrary{arrows,calc}

% Meistens ist \varphi schöner als \phi, genauso bei \theta
\renewcommand{\phi}{\varphi}
\renewcommand{\theta}{\vartheta}

% Aufzählungen anpassen (alternativ: \arabic, \alph)
\renewcommand{\labelenumi}{(\roman{enumi})}

% rwth colors
% colors: blue violet purple carmine red magenta orange yellow grass cyan gold silver
\definecolor{rwth-blue}{cmyk}{1,.5,0,0}\colorlet{rwth-lblue}{rwth-blue!50}\colorlet{rwth-llblue}{rwth-blue!25}
\definecolor{rwth-violet}{cmyk}{.6,.6,0,0}\colorlet{rwth-lviolet}{rwth-violet!50}\colorlet{rwth-llviolet}{rwth-violet!25}
\definecolor{rwth-purple}{cmyk}{.7,1,.35,.15}\colorlet{rwth-lpurple}{rwth-purple!50}\colorlet{rwth-llpurple}{rwth-purple!25}
\definecolor{rwth-carmine}{cmyk}{.25,1,.7,.2}\colorlet{rwth-lcarmine}{rwth-carmine!50}\colorlet{rwth-llcarmine}{rwth-carmine!25}
\definecolor{rwth-red}{cmyk}{.15,1,1,0}\colorlet{rwth-lred}{rwth-red!50}\colorlet{rwth-llred}{rwth-red!25}
\definecolor{rwth-magenta}{cmyk}{0,1,.25,0}\colorlet{rwth-lmagenta}{rwth-magenta!50}\colorlet{rwth-llmagenta}{rwth-magenta!25}
\definecolor{rwth-orange}{cmyk}{0,.4,1,0}\colorlet{rwth-lorange}{rwth-orange!50}\colorlet{rwth-llorange}{rwth-orange!25}
\definecolor{rwth-yellow}{cmyk}{0,0,1,0}\colorlet{rwth-lyellow}{rwth-yellow!50}\colorlet{rwth-llyellow}{rwth-yellow!25}
\definecolor{rwth-grass}{cmyk}{.35,0,1,0}\colorlet{rwth-lgrass}{rwth-grass!50}\colorlet{rwth-llgrass}{rwth-grass!25}
\definecolor{rwth-green}{cmyk}{.7,0,1,0}\colorlet{rwth-lgreen}{rwth-green!50}\colorlet{rwth-llgreen}{rwth-green!25}
\definecolor{rwth-cyan}{cmyk}{1,0,.4,0}\colorlet{rwth-lcyan}{rwth-cyan!50}\colorlet{rwth-llcyan}{rwth-cyan!25}
\definecolor{rwth-teal}{cmyk}{1,.3,.5,.3}\colorlet{rwth-lteal}{rwth-teal!50}\colorlet{rwth-llteal}{rwth-teal!25}
\definecolor{rwth-gold}{cmyk}{.35,.46,.7,.35}
\definecolor{rwth-silver}{cmyk}{.39,.31,.32,.14}

\usepackage{enumitem}
\usepackage[caps]{complexity}

\usepackage{listings}  
\lstset{columns=fullflexible,
        mathescape=true,
        morekeywords={if,else,return,for,while,do,break,def,Input},
        numbers=left,
    	stepnumber=1,
   		showstringspaces=false,
    	tabsize=4,
    	breaklines=true,
    	breakatwhitespace=false,
    	escapeinside={(*}{*)}
        }

% Header i-> inner (bei einseitig links), c -> center, o -> Outer (bei einseitg rechts)
\ihead{BuK WS 2020/21 \\ Tutorium 04 \\\today}
\chead{\Large Übungsblatt 10}
\ohead{Til Mohr, 405959 \\
	   Andrés Montoya, 405409 \\
	   Marc Ludevid Wulf, 405401}
	
\cfoot*{\pagemark} % Seitenzahlen unten

\begin{document}
	\section*{Aufgabe 4}
	\ComplexityFont{QAP} ist in \ComplexityFont{NP}. Hierzu können wir einen Verfiizierer verwenden, welcher als Zertifikat eben ein solches $\sigma$ bekommt und dann überprüft, ob die Aussage gilt. Offensichtlich geht das in polinomieller Zeit.\\
	
	\ComplexityFont{QAP} ist \ComplexityFont{NP}-schwer. Hierzu reduzieren wir \ComplexityFont{Hamiltonkreis} auf \ComplexityFont{QAP}. Für das Hamiltonproblem über Graphen $G=(V,E)$ sein $m=n=\vert V \vert$ und sei $c_{i,j} = 1$, falls es eine Kante zwischen $V_i$ und $V_j$ gibt und $c_{i,j} = n+1$ sonst. Außerdem sei $d_{i,j} = 1$, falls $(i+1) \text{ mod } n = j$, andernfalls $d_{i,j}=0$. Schließlich sei $b=n$. Diese Berechnung ist offensichtlich polynomiell.\\
	
	\ComplexityFont{Hamiltonkreis} $\Rightarrow$ \ComplexityFont{QAP}.\\
	Wenn es einen Hamiltonkreis $H=(v_{\pi(1)},...,v_{\pi(n)})$ gibt, dann gibt es auch für das QuadraticAssignment eine Lösung mit $\sigma = \pi$. Dies liegt daran, dass für diese Permutation gilt, dass alle $d_{\sigma(i),\sigma(j)} = 1$ gilt, dass dann auch $c_{i,j}=1$ ist, und somit die Summe der Multiplikationen von $c$ und $d$ gleich $n$ ist, da alle anderen Werte von $d$ gleich $0$ sind.\\
	
	\ComplexityFont{QAP} $\Rightarrow$ \ComplexityFont{Hamiltonkreis}\\
	Wenn es eine Lösung $\sigma$ für das QuadraticAssignment gibt, gilt für diese Lösung folgendes: $\sigma$ ist eine Permutation über Knoten. Außerdem wissen wir, dass zwischen jeder dieser Knoten eine Kante existieren muss, denn angenommen zwischen zwei Knoten $v_i$ und $v_j$ keine Kante existiert und somit $c_{i,j}=n+1$ ist und $(\pi(i) + 1) \text{ mod } n = \pi(j)$ sei (also $j$ kommt eins nach $i$ in der Permutation), dann wäre also $d_{\pi(i),\pi(j)}=1$ und die Multiplikation beider würde $n+1$ ergeben. Somit könnte die Schranke $b$ nicht mehr eingehalten werden. Eine Permutation der Knoten, die in der Reihenfolge, wie sie in der Permutation stehen, über Kanten verbunden sind, ist ein Hamiltonkreis.\\
	
	Da \ComplexityFont{QAP} sowohl in \ComplexityFont{NP} ist, als auch \ComplexityFont{NP}-schwer ist, ist es \ComplexityFont{NP}-vollständig.

\newpage

	\section*{Aufgabe 5}

	\ComplexityFont{SetCover} ist in \ComplexityFont{NP}. Hierzu können wir einen Verifizierer verwenden, welcher als Zertifikat eine Menge $C$ bekommt. Der Verifizierer überprüft nun, ob $\vert C \vert \leq k$ gilt, $C \subseteq S$ gilt, und anschließend, ob $U = \bigcup_{V \in C} V$ gilt. Offensichtlich läuft dieser Verifizierer in polynomieller Zeit.\\
	
	\ComplexityFont{SetCover} ist \ComplexityFont{NP}-schwer. Hierzu reduzieren wir \ComplexityFont{Hamiltonkreis} auf \ComplexityFont{SetCover}. Für ein Hamiltonproblem über Graphen $G=(V,E)$ und $V=\{v_1,...,v_n\}$ sei $U=\{v_{1_{in}},v_1,v_{1_{out}},...,v_{n_{in}},v_n,v_{n_{out}}\}$. Außerdem sei für jedes $e=(v_i,v_j)$ in $E$ eine Menge $\{v_{i_{out}},v_{j_{in}}\}$ in S. Außerdem sei $k= \frac{\vert U \vert}{2} = \vert V \vert n$. Diese Berechnung ist offensichtlich in polynomieller Zeit möglich.\\
	
	\ComplexityFont{Hamiltonkreis} $\Rightarrow$ \ComplexityFont{SetCover}:\\
	Wenn es einen Hamiltonkreis $H=(v_{\pi(1)},...,v_{\pi(n)})$ gibt, dann gibt es Kanten $E_H=(e_1,...,e_n)$, sodass jede Kante $e_i$ von $v_{\pi(i)}$ nach $v_{\pi(i)+1}$ verläuft. Spezialfall ist $e_n$, dass von $v_{\pi(n)}$ nach $v_{\pi(1)}$ verläuft. Also gibt es in $S$ eine Untermenge $C$, wobei $C_i=\{v_{j_{out}},v_{k_{in}}\}$ der Kante $e_i$ entspricht, die von $v_j$ nach $v_k$ verläuft. Da diese Untermenge genau die Kanten des Hamiltonkreis entsprechen, ist jedes Element $v_{i_{out}}$ und $v_{i_{in}}$ genau ein Mal vertreten und die Kardinalität von $C$ ist genau $k$.\\
	
	\ComplexityFont{SetCover} $\Rightarrow$ \ComplexityFont{Hamiltonkreis}:\\
	Für eine gegebene Lösung $C$ für gegebenes $S$ und $k$ gilt folgendes: Da $k= \frac{\vert U \vert}{2}$, jede Menge in $C$ genau $2$ Elemente hat und die Vereinigung der Mengen in $C$ alle Knoten von $U$ abdeckt, ist jedes Element von $U$ genau ein Mal in $C$ vertreten.\\
	Nun lässt sich ein Hamiltonkreis folgendermaßen konstruieren: Starte bei $v_1$ und suche das $C_i$, dass $v_{1_{out}}$ enthält. In derselben Menge wird ein $v_{i_{in}}$ sein. Dieses $v_i$ ist der nächste Knoten. Wiederhole, bis $v_1$ wieder erreicht wird. Dass das Durchlaufen der Knoten in der so produzierten Reihenfolge auch im Problemgraphen möglich ist, liegt an der Konstruktion von $S$. Zu zeigen ist aber noch, dass zum einen kein Knoten doppelt abgelaufen wird, und dass keiner nicht durchlaufen wird. Das erste stimmt, da wie oben gezeigt wurde, jedes Element von $U$ genau einmal in $C$ vorhanden sein wird, und dementsprechend jeweils nur ein $v_{i_{in}}$ und ein $v_{i_{out}}$ pro Knoten $v_i$ existieren wird. Dass jeder Knoten erreicht wird liegt daran, dass der Kreis $k$ Kanten hat und kein Knoten $2$ Mal durchlaufen wird, also $k$ Knoten durchlaufen werden, und das ist genau die Kardinalität der Knoten in $G$.
	
	Da \ComplexityFont{SetCover} sowohl in \ComplexityFont{NP} ist, als auch \ComplexityFont{NP}-schwer ist, ist es \ComplexityFont{NP}-vollständig.

\newpage

	\section*{Aufgabe 6}
	\ComplexityFont{SetPacking} ist in \ComplexityFont{NP}. Hierzu können wir einen Verifizierer verwenden, welcher als Zertifikat eine Menge $S$ bekommt. Der Verifizierer überprüft nun, ob $\vert S \vert = k$ gilt, $S \subseteq \text{Pot}(U)$ gilt, und anschließend, ob $C_i \cap C_j = \emptyset, i \neq j$ gilt. Offensichtlich läuft dieser Verifizierer in polynomieller Zeit.\\
	
	\ComplexityFont{SetPacking} ist \ComplexityFont{NP}-schwer. Hierzu reduzieren wir \ComplexityFont{IndependentSet} auf \ComplexityFont{SetPacking}. Aus einer Instanz aus \ComplexityFont{IndependentSet} können wir eine Instanz aus \ComplexityFont{SetPacking} wie folgt konstruieren:\\
	$G=(V,E)$ ist ein ungerichteter Graph mit $n$ Knoten. Wir setzen $U = E$, $S = \text{Pot}(U)$ und übernehmen das $k$ aus der \ComplexityFont{IndependentSet}-Instanz in das $k$ der \ComplexityFont{SetPacking}-Instanz. Unser $C \subseteq S$ besteht nun aus Mengen $C_i$, $i \in [\, n ]\,$, wobei jedes $C_i$ alle Kanten aus $E$ enthält, die einen Endknoten am Knoten $i$ haben. Also hat $C$ eine Größe von $n$. Diese Konstruktion können wir offensichtlich in polynomieller Zeit erstellen.\\
	
	\ComplexityFont{IndependentSet} $\Rightarrow$ \ComplexityFont{SetPacking}:\\
	$G$ hat ein \textit{IndependentSet} $S'$ von Größe $\geq k$. Nun führen wir unsere Konstruktion durch und bekommen die Menge $C$ der Größe $n$. Anschließend können wir noch jedes $C_i$ aus $C$ löschen, wenn $i \not\in S'$ ist. Damit hat $C$ danach die Größe $\vert S' \vert$, insbesondere also $\geq k$. Ist $\vert C \vert > k$, so löschen wir zusätzlich noch weitere $C_i$'s aus $C$, bis $\vert C \vert = k$ gilt. Alle verbleibenden $C_i$ in $C$ sind voneinander disjunkt: Alle Elemente in $U$ sind ja Kanten $e$ in $G$. Da $S'$ ein \textit{IndependentSet} ist, ist höchstens einer der beiden Endpunkte von $e$ in $S'$. Hat $e$ keinen Endpunkt in $S'$, gibt es auch kein $C_i$ in $C$, welches $e$ besitzt. Hat $e$ jedoch einen Endpunkt $j$ in $S'$, ist aber $C_j$ nicht mehr in $C$, dann ist auch $e$ nicht mehr in $C$ vorhanden. Hat $e$ aber einen Endpunkt $j$ in $S'$ und ist $C_j$ noch in $C$, dann ist $e$ auch durch $C_j$ in $C$ vertreten. Jedoch kann $e$ nicht noch durch ein anderes $C_i$ in $C$ vertreten sein, da ja $S'$ ein \textit{IndependentSet} ist. Also gilt $C_i \cap C_j, i \neq j$.\\
	
	\ComplexityFont{SetPacking} $\Rightarrow$ \ComplexityFont{IndependentSet}:\\
	Sei $C$ ein \textit{SetPacking} der Größe $k$ nach unserem Konstrukt mit $C_i \cap C_j, i \neq j$. Erstelle nun aus $C$ (bzw. den $C_i$) eine Menge an Knoten $S'$. $i \in [\, n ]\,$ ist genau dann in $S'$, falls $C_i$ in $C$ ist. Damit gilt $\vert S' \vert = \vert C \vert = k \geq k$. Nun betrachte jede Kante $e$ aus $U$. Da $C$ das \textit{SetPacking} von $U$ ist, gibt es höchstens eine Menge $C_i$ in $C$, die $e$ enthält. Unserer Konstruktion nach besitzen die $C_i$'s nur Kanten, die als Endknoten den Knoten $i$ haben. Ist also $e$ in einem $C_i$ vertreten, muss $S'$ $i$ besitzen. Damit gilt, dass $U'$ ein \textit{IndependentSet} ist.\\
	
	Da \ComplexityFont{SetPacking} sowohl in \ComplexityFont{NP} ist, als auch \ComplexityFont{NP}-schwer ist, ist es \ComplexityFont{NP}-vollständig.
	
\end{document}

