\documentclass[a4paper,11pt]{scrartcl}
\usepackage[a4paper, left=2cm, right=4.5cm, top=2cm, bottom=2cm]{geometry} % kleinere Ränder

%Paket für Header in Koma-Klassen (scrartcl, scrrprt, scrbook, scrlttr2)
\usepackage[headsepline]{scrlayer-scrpage}
% Header groß genug für 3 Zeilen machen
\setlength{\headheight}{3\baselineskip}


% Default Header löschen
\pagestyle{scrheadings}
\clearpairofpagestyles

% nicht kursiv gedruckte header
\setkomafont{pagehead}{\sffamily\upshape}

% Links im Dokuement sowie \url schön machen
\usepackage[colorlinks,pdfpagelabels,pdfstartview = FitH, bookmarksopen = true,bookmarksnumbered = true, linkcolor = black, plainpages = false, hypertexnames = false, citecolor = black]{hyperref}

% Umlaute in der Datei erlauben, auf deutsch umstellen
\usepackage[utf8]{inputenc}
\usepackage[ngerman]{babel}

% Mathesymbole und Ähnliches
\usepackage{amsmath}
\usepackage{mathtools}
\usepackage{amssymb}
\usepackage{microtype}
\newcommand{\NN}{{\mathbb N}}
\newcommand{\RR}{{\mathbb R}}
\newcommand{\QQ}{{\mathbb Q}}
\newcommand{\ZZ}{{\mathbb{Z}}}

% Komplexitätsklassen
\newcommand{\pc}{\ensuremath{{\sf P}}}
\newcommand{\np}{\ensuremath{{\sf NP}}}
\newcommand{\npc}{\ensuremath{{\sf NPC}}}
\newcommand{\pspace}{\ensuremath{{\sf PSPACE}}}
\newcommand{\exptime}{\ensuremath{{\sf EXPTIME}}}
\newcommand{\CClassNP}{\textup{NP}\xspace}
\newcommand{\CClassP}{\textup{P}\xspace}

% Weitere pakete
\usepackage{multicol}
\usepackage{booktabs}

% Abbildungen
\usepackage{tikz}
\usetikzlibrary{arrows,calc}

% Meistens ist \varphi schöner als \phi, genauso bei \theta
\renewcommand{\phi}{\varphi}
\renewcommand{\theta}{\vartheta}

% Aufzählungen anpassen (alternativ: \arabic, \alph)
\renewcommand{\labelenumi}{(\roman{enumi})}

% rwth colors
% colors: blue violet purple carmine red magenta orange yellow grass cyan gold silver
\definecolor{rwth-blue}{cmyk}{1,.5,0,0}\colorlet{rwth-lblue}{rwth-blue!50}\colorlet{rwth-llblue}{rwth-blue!25}
\definecolor{rwth-violet}{cmyk}{.6,.6,0,0}\colorlet{rwth-lviolet}{rwth-violet!50}\colorlet{rwth-llviolet}{rwth-violet!25}
\definecolor{rwth-purple}{cmyk}{.7,1,.35,.15}\colorlet{rwth-lpurple}{rwth-purple!50}\colorlet{rwth-llpurple}{rwth-purple!25}
\definecolor{rwth-carmine}{cmyk}{.25,1,.7,.2}\colorlet{rwth-lcarmine}{rwth-carmine!50}\colorlet{rwth-llcarmine}{rwth-carmine!25}
\definecolor{rwth-red}{cmyk}{.15,1,1,0}\colorlet{rwth-lred}{rwth-red!50}\colorlet{rwth-llred}{rwth-red!25}
\definecolor{rwth-magenta}{cmyk}{0,1,.25,0}\colorlet{rwth-lmagenta}{rwth-magenta!50}\colorlet{rwth-llmagenta}{rwth-magenta!25}
\definecolor{rwth-orange}{cmyk}{0,.4,1,0}\colorlet{rwth-lorange}{rwth-orange!50}\colorlet{rwth-llorange}{rwth-orange!25}
\definecolor{rwth-yellow}{cmyk}{0,0,1,0}\colorlet{rwth-lyellow}{rwth-yellow!50}\colorlet{rwth-llyellow}{rwth-yellow!25}
\definecolor{rwth-grass}{cmyk}{.35,0,1,0}\colorlet{rwth-lgrass}{rwth-grass!50}\colorlet{rwth-llgrass}{rwth-grass!25}
\definecolor{rwth-green}{cmyk}{.7,0,1,0}\colorlet{rwth-lgreen}{rwth-green!50}\colorlet{rwth-llgreen}{rwth-green!25}
\definecolor{rwth-cyan}{cmyk}{1,0,.4,0}\colorlet{rwth-lcyan}{rwth-cyan!50}\colorlet{rwth-llcyan}{rwth-cyan!25}
\definecolor{rwth-teal}{cmyk}{1,.3,.5,.3}\colorlet{rwth-lteal}{rwth-teal!50}\colorlet{rwth-llteal}{rwth-teal!25}
\definecolor{rwth-gold}{cmyk}{.35,.46,.7,.35}
\definecolor{rwth-silver}{cmyk}{.39,.31,.32,.14}

\usepackage{enumitem}
\usepackage[caps]{complexity}

\usepackage{listings}  
\lstset{columns=fullflexible,
        mathescape=true,
        morekeywords={if,else,return,while,do,break,Input},
        numbers=left,
    	stepnumber=1,
   		showstringspaces=false,
    	tabsize=4,
    	breaklines=true,
    	breakatwhitespace=false,
    	escapeinside={(*}{*)}
        }

% Header i-> inner (bei einseitig links), c -> center, o -> Outer (bei einseitg rechts)
\ihead{BuK WS 2020/21 \\ Tutorium 04 \\\today}
\chead{\Large Übungsblatt 09}
\ohead{Til Mohr, 405959 \\
	   Andrés Montoya, 405409 \\
	   Marc Ludevid Wulf, 405401}
	
\cfoot*{\pagemark} % Seitenzahlen unten

\begin{document}
	\section*{Aufgabe 4}
	\begin{lstlisting}
	Input G=(V,E)	
	
	$v_{start}$ $\coloneqq$ getFirst(V)
	K $\coloneqq$ $\{v_{start}\}$
	E' $\coloneqq$ E
	weights $\coloneqq$ sumAllWeights(E')
	tourWeights $\coloneqq$ 0
	
	while !SetEquals(K,V):
		u $\coloneqq$ getLast(K)
		E'' $\coloneqq$ E'
		
		while true:
			if(E'' = $\emptyset$):
				return $\emptyset$ //Error
			
			
			weights' $\coloneqq$ weights
			tourWeights' $\coloneqq$ tourWeights
			E''' $\coloneqq$ E'
			
			v $\coloneqq$ getMinCostFrom(E'', u)
			cost $\coloneqq$ getCost(E'', (u,v))
			weights' -= cost
			tourWeights' += cost
			
			forall(k in K && k $\neq$ $v_{start}$):
				E''' = removeIfExists(E''', (k,v))
			
			if((*\ComplexityFont{TSP-E}*) ((V',E'''), (tourWeights' + getCost(E'',(v,$v_start$))))
				return add(K,v)
			else if((*\ComplexityFont{TSP-E}*) ((V',E'''),weights))
				K = add(K,v)
				E' = E'''
				weights = weights'
				tourWeights = tourWeights'
				break
			else
				E'' = remove(E'', (u,v))
	
	
	return K
			
	\end{lstlisting}
	
TODO: Korrektheit + Laufzeit	
	
	\section*{Aufgabe 5}
	Hierzu nehmen wir uns eine Formel $\phi$ in 3-KNF mit den Klauseln $k_i$.
	Jedes $k_i$ in $\phi$ ist daher der Form $(x_i \lor y_i \lor z_i)$. Nun teilen wir jedes $k_i$ in zwei weitere Klauseln $k_i'$ und $k_i''$, wobei wir auch noch für jede Klausel eine Variable $c_i$, und für alle Klauseln die Variable $f = 0$ einführen.\\
	Aus $k_i$ wird also $k_i' \land k_i'' = (x_i \lor y_i \lor c_i) \land (\bar{c_i} \lor z_i \lor f)$.
	Hierbei wird $c_i$ auf $\neg (x_i \lor y_i)$ gesetzt.\\
	Diese Konstruktion können wir in Linearer Zeit erstellen.\\
	
	Korrektheit:\\
		Sei $\phi$ in 3-KNF mit gegebener Belegung der Variablen. Dann ist jede Klausel $k_i$ der Form $(x_i \lor y_i \lor z_i)$.\\
		Hieraus konstruieren wir eine neue aussagenlogische Formel $\phi '$ wie oben beschrieben.\\
		\begin{itemize}
		\item Falls $x_i$ oder $y_i$ wahr ist, so setzen wir $c_i$ auf $0$. Damit ist zum einen die Klausel $k_i'$ wahr, zum anderen gibt es in dieser Klausel mindestens sowohl ein wahres, als auch ein falsches Literal. Desweiteren ist dadurch (unabhängig von $z_i$) auch $k_i''$ wahr, da dort $\bar{c_i}$ enthalten ist. Auch diese Klausel besitzt sowohl ein wahres, als auch ein falsches Literal ($f$).
		
		\item Falls $x_i$ und $y_i$ falsch sind, aber $z_1$ wahr, dann setzen wir $c_i$ auf $1$. Damit ist die erste Klausel ($k_i'$) wahr und besitzt auch wieder mindestens ein wahres, als auch ein falsches Literal. Die Klausel $k_i'$ ist aufgrund von $z_i$ auch wahr, und besitzt auch wieder ein wahres und ein falsches Literal ($f$).
		
		\item Falls keines der drei Literale wahr ist, so gibt es ja keine Erfüllende Belegung der Variablen, weshalb die gesamte Formel nicht in \ComplexityFont{3-SAT} ist. Trotzdem setzen wir hier ja $c_i$ auf $1$, weshalb ja die Klausel $k_i'$ war ist. Jedoch ist die Klausel $k_i''$ folglich nicht wahr, weshalb die resultierende Formel auch nicht in \ComplexityFont{NOT-ALL-EQUAL-SAT} ist.
		\end{itemize}
		Damit ist der Wahrheitsgrad von $\phi '$ immer genau derselbe wie von $\phi$ für jede Belegung.\\
		Hat also $\phi$ keine erfüllende Belegung (nicht in \ComplexityFont{3-SAT}), so hat auch $\phi '$ keine erfüllende Belegung (nicht in \ComplexityFont{NOT-ALL-EQUAL-SAT}).\\
		Hat aber $\phi$ eine erfüllende Belegung (ist in \ComplexityFont{3-SAT}), so hat $\phi '$ eine erfüllende Belegung, die die gleichen Variablen gleich belegt, und weitere Variablen so belegt, dass jede Klausel mindestens ein wahres und ein falsches Literal besitzt (ist in \ComplexityFont{NOT-ALL-EQUAL-SAT}).
	
	\section*{Aufgabe 6}
	\begin{enumerate}[label=(\alph*)]
	\item
		Hierzu bauen wir uns einen Verifizierer, welches als Zertifikat einen Vektor $x \in \{0,1\}^n$ übergeben bekommt, welches einfach codiert werden kann als $x_1 ... x_n$.\\
		Unser Verifizierer überprüft nun Folgende Dinge:
		\begin{itemize}
		\item	Eingabe korrekt formatiert (Linearer Zeitaufwand in $n$).
		\item	Überstimmt die Vektordimension $n$ von $x$ überein mit der gegebenen Matrix $A$ und dem Vektor $b$? (höchstens Quadratischer Zeitaufwand in $n \cdot m$, je nach Codierung der Matrix $A$)
		\item	Berechnung von $Ax$ (Quadratischer Zeitaufwand in $n \cdot m$)
		\item	Abgleich, ob $Ax \geq b$ (Linearer Zeitaufwand in $n$).
		\end{itemize}
		Damit läuft der Verifizierer in polynomieller Zeit und \ComplexityFont{$\{-1,0,1\}$-RESTRICTED INTEGER PROGRAMMING} ist in \ComplexityFont{NP}.
		
	\item 
	\end{enumerate}
	
\end{document}

