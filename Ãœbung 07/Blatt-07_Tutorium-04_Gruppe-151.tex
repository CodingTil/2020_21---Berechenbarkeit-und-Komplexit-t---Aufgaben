\documentclass[a4paper,11pt]{scrartcl}
\usepackage[a4paper, left=2cm, right=4.5cm, top=2cm, bottom=2cm]{geometry} % kleinere Ränder

%Paket für Header in Koma-Klassen (scrartcl, scrrprt, scrbook, scrlttr2)
\usepackage[headsepline]{scrlayer-scrpage}
% Header groß genug für 3 Zeilen machen
\setlength{\headheight}{3\baselineskip}


% Default Header löschen
\pagestyle{scrheadings}
\clearpairofpagestyles

% nicht kursiv gedruckte header
\setkomafont{pagehead}{\sffamily\upshape}

% Links im Dokuement sowie \url schön machen
\usepackage[colorlinks,pdfpagelabels,pdfstartview = FitH, bookmarksopen = true,bookmarksnumbered = true, linkcolor = black, plainpages = false, hypertexnames = false, citecolor = black]{hyperref}

% Umlaute in der Datei erlauben, auf deutsch umstellen
\usepackage[utf8]{inputenc}
\usepackage[ngerman]{babel}

% Mathesymbole und Ähnliches
\usepackage{amsmath}
\usepackage{mathtools}
\usepackage{amssymb}
\usepackage{microtype}
\newcommand{\NN}{{\mathbb N}}
\newcommand{\RR}{{\mathbb R}}
\newcommand{\QQ}{{\mathbb Q}}
\newcommand{\ZZ}{{\mathbb{Z}}}

% Komplexitätsklassen
\newcommand{\pc}{\ensuremath{{\sf P}}}
\newcommand{\np}{\ensuremath{{\sf NP}}}
\newcommand{\npc}{\ensuremath{{\sf NPC}}}
\newcommand{\pspace}{\ensuremath{{\sf PSPACE}}}
\newcommand{\exptime}{\ensuremath{{\sf EXPTIME}}}
\newcommand{\CClassNP}{\textup{NP}\xspace}
\newcommand{\CClassP}{\textup{P}\xspace}

% Weitere pakete
\usepackage{multicol}
\usepackage{booktabs}

% Abbildungen
\usepackage{tikz}
\usetikzlibrary{arrows,calc}

% Meistens ist \varphi schöner als \phi, genauso bei \theta
\renewcommand{\phi}{\varphi}
\renewcommand{\theta}{\vartheta}

% Aufzählungen anpassen (alternativ: \arabic, \alph)
\renewcommand{\labelenumi}{(\roman{enumi})}

% rwth colors
% colors: blue violet purple carmine red magenta orange yellow grass cyan gold silver
\definecolor{rwth-blue}{cmyk}{1,.5,0,0}\colorlet{rwth-lblue}{rwth-blue!50}\colorlet{rwth-llblue}{rwth-blue!25}
\definecolor{rwth-violet}{cmyk}{.6,.6,0,0}\colorlet{rwth-lviolet}{rwth-violet!50}\colorlet{rwth-llviolet}{rwth-violet!25}
\definecolor{rwth-purple}{cmyk}{.7,1,.35,.15}\colorlet{rwth-lpurple}{rwth-purple!50}\colorlet{rwth-llpurple}{rwth-purple!25}
\definecolor{rwth-carmine}{cmyk}{.25,1,.7,.2}\colorlet{rwth-lcarmine}{rwth-carmine!50}\colorlet{rwth-llcarmine}{rwth-carmine!25}
\definecolor{rwth-red}{cmyk}{.15,1,1,0}\colorlet{rwth-lred}{rwth-red!50}\colorlet{rwth-llred}{rwth-red!25}
\definecolor{rwth-magenta}{cmyk}{0,1,.25,0}\colorlet{rwth-lmagenta}{rwth-magenta!50}\colorlet{rwth-llmagenta}{rwth-magenta!25}
\definecolor{rwth-orange}{cmyk}{0,.4,1,0}\colorlet{rwth-lorange}{rwth-orange!50}\colorlet{rwth-llorange}{rwth-orange!25}
\definecolor{rwth-yellow}{cmyk}{0,0,1,0}\colorlet{rwth-lyellow}{rwth-yellow!50}\colorlet{rwth-llyellow}{rwth-yellow!25}
\definecolor{rwth-grass}{cmyk}{.35,0,1,0}\colorlet{rwth-lgrass}{rwth-grass!50}\colorlet{rwth-llgrass}{rwth-grass!25}
\definecolor{rwth-green}{cmyk}{.7,0,1,0}\colorlet{rwth-lgreen}{rwth-green!50}\colorlet{rwth-llgreen}{rwth-green!25}
\definecolor{rwth-cyan}{cmyk}{1,0,.4,0}\colorlet{rwth-lcyan}{rwth-cyan!50}\colorlet{rwth-llcyan}{rwth-cyan!25}
\definecolor{rwth-teal}{cmyk}{1,.3,.5,.3}\colorlet{rwth-lteal}{rwth-teal!50}\colorlet{rwth-llteal}{rwth-teal!25}
\definecolor{rwth-gold}{cmyk}{.35,.46,.7,.35}
\definecolor{rwth-silver}{cmyk}{.39,.31,.32,.14}

\usepackage{enumitem}

\usepackage{listings}  
\lstset{columns=fullflexible,
        mathescape=true,
        morekeywords={if,else,return,while,Input}
        }

% Header i-> inner (bei einseitig links), c -> center, o -> Outer (bei einseitg rechts)
\ihead{BuK WS 2020/21 \\ Tutorium 04 \\\today}
\chead{\Large Übungsblatt 07}
\ohead{Til Mohr, 405959 \\
	   Andrés Montoya, 405409 \\
	   Marc Ludevid Wulf, 405401}
	
\cfoot*{\pagemark} % Seitenzahlen unten

\begin{document}
	\section*{Aufgabe 4}
	\begin{lstlisting}
	$x_3$ := $x_4$ + 0;				// Set $x_3$ to 0
	$x_0$ := $x_4$ + 1;				// Set $x_0$ to 1
	// Note 1
	LOOP $x_2$ DO
		$x_3$ := $x_4$ + 0; 		// Set $x_3$ to 0
		// Note 2	
		LOOP $x_1$ DO
			LOOP $x_0$ DO
				$x_3$ := $x_3$ + 1
			END
		END;
		$x_0$ := $x_3$ + 0
	END
	\end{lstlisting}
	Unser Programm berechnet für $x_1^{x_2}$ $x_2$-mal $1 \cdot x_1 \cdot ... \cdot x_1$.\\
	Hierzu müssen wir also $x_2$ mal eine Multiplikation $x_0 \coloneqq x_0 \cdot x_1$ durchführen (\verb|Note 1|). Dazu initialisieren wir $x_0$ auf $1$.\\
	Die Multiplikation selber müssen wir nochmal in 2 Schleifen aufteilen (\verb|Note 2|): Wir setzten unsere Hilfsvariable $x_3$ auf $0$, addieren dann $x_1 \cdot x_0$-mal eine $1$ auf $x_3$, und speichern dann das Ergebnis in $x_0$.
	
	
	\section*{Aufgabe 5}
	\begin{enumerate}[label=\alph*)]
	\item	Unser Programm benutzt Variablen zusätzlich zu den aus \verb|P| und \verb|Q|, die die Programme selber nicht benutzen. Diese Variablen sind eben $x_l$, eine Hilfsvariable $x_m$ und $x_{k+1}$, welche immer $0$ ist.
	\begin{lstlisting}
	$x_m$ := $x_{k+1}$ + 1;		// Set $x_m$ to 1
	WHILE $x_l$ $\neq$ 0 DO
		Q;
		$x_l$ := $x_{k+1}$ + 0;	// Set $x_l$ to 0
		$x_m$ := $x_{k+1}$ + 0	// Set $x_m$ to 0
	END;
	WHILE $x_m$ $\neq$ 0 DO
		P;
		$x_l$ := $x_{k+1}$ + 0;	// Set $x_l$ to 0
		$x_m$ := $x_{k+1}$ + 0	// Set $x_m$ to 0
	END
	\end{lstlisting}
	Unser Programm verwendet die Variable $x_m$ als eine Art Mutex. Haben wir nämlich einmal \verb|Q| ausgeführt, können wir \verb|P| nicht ausführen.
	
	\newpage	
	
	\item	Unser Programm benutzt Variablen zusätzlich zu denen aus \verb|P_i|, die die Programme selber nicht benutzen. Diese Variablen sind eben $x_{k+1}$, eine Hilfsvariable $x_m$ und $x_l$, welche immer $0$ ist.
	\begin{lstlisting}
	$x_{k+1}$ := $x_{k+1}$ + 1;		// Increase $x_{k+1}$ by 1
	$x_m$ := $x_l$ + 1;				// Set $x_m$ to 1
	
	// Loop	
	
	WHILE $x_{k+1}$ $\neq$ 0 DO
		$x_{k+1}$ := $x_{k+1}$ - 1;	// Decrease $x_{k+1}$ by 1
		
		
		<---->>>				// Weitere Faelle
		
		
		WHILE $x_m$ $\neq$ 0 DO
			$P_0$;
			$x_m$ := $x_l$ + 0		// Set $x_m$ to 0
		END
	END
	\end{lstlisting}
	Unser Programm verwendet die Variable $x_m$ als eine Art Mutex. Haben wir nämlich einmal ein Programm \verb|P_i| ausgeführt, so können wir kein anderes Programm mehr ausführen.
	Hierbei ersetzt man \verb|<---->>>| so häufig mit der Äußeren Schleife (\verb|P_i| anpassen!), wie es Programme gibt. Die Schleife, die quasi $x_{k+1} = n$ überprüft (wobei $n$ der höchste Index der Programme ist) noch eine Sonderschleife bekommt.
	Beispiel für Programme \verb|P_0, P_1, P_2|:
	\begin{lstlisting}
	$x_{k+1}$ := $x_{k+1}$ + 1;		// Increase $x_{k+1}$ by 1
	$x_m$ := $x_l$ + 1;				// Set $x_m$ to 1
	
	// Loop	
	
	WHILE $x_{k+1}$ $\neq$ 0 DO
		$x_{k+1}$ := $x_{k+1}$ - 1;	// Decrease $x_{k+1}$ by 1
		
		
		WHILE $x_{k+1}$ $\neq$ 0 DO
			$x_{k+1}$ := $x_{k+1}$ - 1;	// Decrease $x_{k+1}$ by 1
		
		
			WHILE $x_{k+1}$ $\neq$ 0 DO
				$x_{k+1}$ := $x_{k+1}$ - 1;	// Decrease $x_{k+1}$ by 1
				
				WHILE $x_{k+1}$ $\neq$ 0 DO
					$x_m$ := $x_l$ + 0		// Set $x_m$ to 0
				END
		
				WHILE $x_m$ $\neq$ 0 DO
					$P_2$;
					$x_m$ := $x_l$ + 0		// Set $x_m$ to 0
				END
			END
		
		
			WHILE $x_m$ $\neq$ 0 DO
				$P_1$;
				$x_m$ := $x_l$ + 0		// Set $x_m$ to 0
			END
		END
		
		
		WHILE $x_m$ $\neq$ 0 DO
			$P_0$;
			$x_m$ := $x_l$ + 0		// Set $x_m$ to 0
		END
	END
	\end{lstlisting}
	\end{enumerate}
	
	
	\newpage
	\section*{Aufgabe 6}
	100-VARIABLE-WHILE Programme sind Turing-mächtig, da sie TMs simulieren können.
	O.B.d.A: Wir betrachten nun eine eine TM mit dem Bandalphabet {0,1,B} und k Zuständen wobei der Zustand 0 der Endzustand ist.

	Um die TM zu simulieren benötigt ein WHILE Programm folgende Variablen:
	$x_0$ als Zustand
	$x_1$ als Wort vor dem Lesekopf
	$x_2$ als Wort nach dem Lesekopf
	$x_3$ als Hilfsvariable für die if-Statements

	Das Programm sieht dann folgendermassen aus:
	Eine äussere While-Schleife durchläuft das Programm solange $x_0$ nicht 0 (Endzustand) ist.
	In jedem Schleifendurchlauf wird mit k if-Statements geprüft in welchem Zustand sich die TM befindet und mit einem weiteren if-Statement geprüft was unter dem Lesekopf steht.
	Dann wird dementsprechend das Band und der Zustand erneuert.
	Die if-Statements für die Zustände brauchen eine Hilfsvariable (siehe Aufgabe 5.b))
	Das Updaten des Zustands geht ohne weitere Hilfsvariablen.
	Für das Updaten des Bandes wird ebenfalls keine Hilfsvariablen benötigt.

	Damit kann eine 100-VARIABLE-WHILE Programm die TM locker simulieren und die 100-VARIABLE-WHILE Programme sind Turing-mächtig.
\end{document}

