\documentclass[a4paper,11pt]{scrartcl}
\usepackage[a4paper, left=2cm, right=4.5cm, top=2cm, bottom=2cm]{geometry} % kleinere Ränder

%Paket für Header in Koma-Klassen (scrartcl, scrrprt, scrbook, scrlttr2)
\usepackage[headsepline]{scrlayer-scrpage}
% Header groß genug für 3 Zeilen machen
\setlength{\headheight}{3\baselineskip}


% Default Header löschen
\pagestyle{scrheadings}
\clearpairofpagestyles

% nicht kursiv gedruckte header
\setkomafont{pagehead}{\sffamily\upshape}

% Links im Dokuement sowie \url schön machen
\usepackage[colorlinks,pdfpagelabels,pdfstartview = FitH, bookmarksopen = true,bookmarksnumbered = true, linkcolor = black, plainpages = false, hypertexnames = false, citecolor = black]{hyperref}

% Umlaute in der Datei erlauben, auf deutsch umstellen
\usepackage[utf8]{inputenc}
\usepackage[ngerman]{babel}

% Mathesymbole und Ähnliches
\usepackage{amsmath}
\usepackage{mathtools}
\usepackage{amssymb}
\usepackage{microtype}
\newcommand{\NN}{{\mathbb N}}
\newcommand{\RR}{{\mathbb R}}
\newcommand{\QQ}{{\mathbb Q}}
\newcommand{\ZZ}{{\mathbb{Z}}}

% Komplexitätsklassen
\newcommand{\pc}{\ensuremath{{\sf P}}}
\newcommand{\np}{\ensuremath{{\sf NP}}}
\newcommand{\npc}{\ensuremath{{\sf NPC}}}
\newcommand{\pspace}{\ensuremath{{\sf PSPACE}}}
\newcommand{\exptime}{\ensuremath{{\sf EXPTIME}}}
\newcommand{\CClassNP}{\textup{NP}\xspace}
\newcommand{\CClassP}{\textup{P}\xspace}

% Weitere pakete
\usepackage{multicol}
\usepackage{booktabs}

% Abbildungen
\usepackage{tikz}
\usetikzlibrary{arrows,calc}

% Meistens ist \varphi schöner als \phi, genauso bei \theta
\renewcommand{\phi}{\varphi}
\renewcommand{\theta}{\vartheta}

% Aufzählungen anpassen (alternativ: \arabic, \alph)
\renewcommand{\labelenumi}{(\roman{enumi})}

% rwth colors
% colors: blue violet purple carmine red magenta orange yellow grass cyan gold silver
\definecolor{rwth-blue}{cmyk}{1,.5,0,0}\colorlet{rwth-lblue}{rwth-blue!50}\colorlet{rwth-llblue}{rwth-blue!25}
\definecolor{rwth-violet}{cmyk}{.6,.6,0,0}\colorlet{rwth-lviolet}{rwth-violet!50}\colorlet{rwth-llviolet}{rwth-violet!25}
\definecolor{rwth-purple}{cmyk}{.7,1,.35,.15}\colorlet{rwth-lpurple}{rwth-purple!50}\colorlet{rwth-llpurple}{rwth-purple!25}
\definecolor{rwth-carmine}{cmyk}{.25,1,.7,.2}\colorlet{rwth-lcarmine}{rwth-carmine!50}\colorlet{rwth-llcarmine}{rwth-carmine!25}
\definecolor{rwth-red}{cmyk}{.15,1,1,0}\colorlet{rwth-lred}{rwth-red!50}\colorlet{rwth-llred}{rwth-red!25}
\definecolor{rwth-magenta}{cmyk}{0,1,.25,0}\colorlet{rwth-lmagenta}{rwth-magenta!50}\colorlet{rwth-llmagenta}{rwth-magenta!25}
\definecolor{rwth-orange}{cmyk}{0,.4,1,0}\colorlet{rwth-lorange}{rwth-orange!50}\colorlet{rwth-llorange}{rwth-orange!25}
\definecolor{rwth-yellow}{cmyk}{0,0,1,0}\colorlet{rwth-lyellow}{rwth-yellow!50}\colorlet{rwth-llyellow}{rwth-yellow!25}
\definecolor{rwth-grass}{cmyk}{.35,0,1,0}\colorlet{rwth-lgrass}{rwth-grass!50}\colorlet{rwth-llgrass}{rwth-grass!25}
\definecolor{rwth-green}{cmyk}{.7,0,1,0}\colorlet{rwth-lgreen}{rwth-green!50}\colorlet{rwth-llgreen}{rwth-green!25}
\definecolor{rwth-cyan}{cmyk}{1,0,.4,0}\colorlet{rwth-lcyan}{rwth-cyan!50}\colorlet{rwth-llcyan}{rwth-cyan!25}
\definecolor{rwth-teal}{cmyk}{1,.3,.5,.3}\colorlet{rwth-lteal}{rwth-teal!50}\colorlet{rwth-llteal}{rwth-teal!25}
\definecolor{rwth-gold}{cmyk}{.35,.46,.7,.35}
\definecolor{rwth-silver}{cmyk}{.39,.31,.32,.14}

\usepackage{enumitem}
\usepackage[caps]{complexity}

\usepackage{listings}  
\lstset{columns=fullflexible,
        mathescape=true,
        morekeywords={if,else,return,for,while,do,break,def,Input},
        numbers=left,
    	stepnumber=1,
   		showstringspaces=false,
    	tabsize=4,
    	breaklines=true,
    	breakatwhitespace=false,
    	escapeinside={(*}{*)}
        }

% Header i-> inner (bei einseitig links), c -> center, o -> Outer (bei einseitg rechts)
\ihead{BuK WS 2020/21 \\ Tutorium 04 \\\today}
\chead{\Large Übungsblatt 09}
\ohead{Til Mohr, 405959 \\
	   Andrés Montoya, 405409 \\
	   Marc Ludevid Wulf, 405401}
	
\cfoot*{\pagemark} % Seitenzahlen unten

\begin{document}
	\section*{Aufgabe 4}
	\begin{lstlisting}
	Input G=(V,E)	
	
	// Get optimal tour cost
	def getTourCostOpt(G=(V,E)):
		n = size(V)
		E = sort(E)
		E' = getLast(E,n) // Returns Set with last n elements in E
		opt = sumWeights(E')
		while true:
			b = opt - 1
			if (*\ComplexityFont{TSP-E}*)(G,b)):
				opt = b
			else:
				break
		return opt
	
	// Calc Graph, which only contains the tour
	opt = getTourCostOpt(G)
	for $e \in E$:
		G' = (V, remove(E,e))
		if (*\ComplexityFont{TSP-E}*)(G',opt):
			G = G'
	
	// Calulate the path
	$v_{start}$ = getFirst(V)
	K = $\{v_{start}\}$
	e = getRandomEdgeFrom(E, $v_{start}$)
	E = remove(E, e)
	while true:
		u = getNeighbourOf(E, getLast(K)) // get the only neighbour
		K = add(K, u) // add u to K
	
	
	return K
	\end{lstlisting}
	
	Laufzeit:\\
	Die Methode \verb|getTourCostOpt(G)| ist offensichtlich in polynomieller Zeit berechenbar. Das sortieren von E geht in quadratischer Zeit, das Lesen der letzten $n$ Elemente in linearer Zeit, genauso das Aufsummieren der Kosten von den Kanten ein \verb|E'|. Die Schleife in dieser Methode wird auch nur höchstens \verb|opt| mal ausgeführt (abhängig von den Kantengewichten), da wir \verb|opt| mit jedem Schleifendurchlauf dekrementieren. \ComplexityFont{TSP-E}\verb|(G,b)| können wir ja auch in polynomieller Zeit berechnen.\\
	Die Schleife im 2. Teil des Algorithmus wird für jede Kante in \verb|E| ausgeführt. Also kann es nur maximal $n \cdot n$ Schleifendurchläufe geben. Offensichtlich ist auch jeder Schleifendurchlauf in polynomieller Zeit berechenbar.\\
	Die Schleife im 3. Teil des Algorithmus wird genau $n$ mal ausgeführt, da wir im Prinzip über jeden Knoten in \verb|V| iterieren.\\
	Damit ist der gesamte Algorithmus in polynomieller Zeit berechenbar.\\
	
	Korrektheit:\\
	Die Methode bestimmt für einen übergebenen Graphen \verb|G| die Optimale Tour Kosten des \ComplexityFont{TSP}. Da jede Tour in \verb|G| genau $n$ viele Knoten besitzt, läuft diese Tour dann auch über genau $n$ vielen Kanten aus \verb|E|. Da die Tour also höchstens über die $n$ schwersten Kanten laufen kann, setzen wir unsere optimale Kosten auf die Summe der Kosten dieser Kanten. Nun überprüfen wir, ob es auch Touren gibt mit geringeren Gesamtkosten. Dazu könnten wir natürlich jede mögliche $n$-Kombination an Kanten überprüfen, doch es ist viel simpler, unsere bisherigen optimalen Kosten zu verringern (möglich, da alle Kosten in $\mathbb{N}$ liegen). So können wir schrittweise die Optimalen Tourkosten von \verb|G| bestimmen.\\
	In dem 2. Teil des Algorithmus bestimmen wir mithilfe der obigen Methode die optimalen Kosten von einer Tour in \verb|G|. Anschließend überprüfen wir für jede Kante in \verb|E|, ob diese für die Optimale Tour relevant ist. Falls nicht, so löschen wir diese. Mit diesem Teil minimieren wir also den Graphen \verb|G|, sodass dieser nur noch die Kanten der optimalen Tour enthält.\\
	Im letzten Teil müssen wir noch eine Reihenfolge der Knoten finden, in der die optimale Tour laufen kann. Dazu bestimmen wir uns einen Startknoten $v_{start}$, und gehen dann die Kanten der Knoten nacheinander entlang.\\
	Also können wir mithilfe von \ComplexityFont{TSP-E} auch \ComplexityFont{TSP} bestimmen.
	
\newpage	
	
	\section*{Aufgabe 5}
	Hierzu nehmen wir uns eine Formel $\phi$ in 3-KNF mit den Klauseln $k_i$.
	Jedes $k_i$ in $\phi$ ist daher der Form $(x_i \lor y_i \lor z_i)$. Nun teilen wir jedes $k_i$ in zwei weitere Klauseln $k_i'$ und $k_i''$, wobei wir auch noch für jede Klausel eine Variable $c_i$, und für alle Klauseln eine weitere Variable $f$ einführen.\\
	Aus $k_i$ wird also $k_i' \land k_i'' = (x_i \lor y_i \lor c_i) \land (\bar{c_i} \lor z_i \lor f)$.
	Diese Konstruktion können wir in Linearer Zeit erstellen.\\
	
	Korrektheit:\\
	
	\ComplexityFont{3-SAT} $\Rightarrow$ \ComplexityFont{NOT-ALL-EQUAL-SAT}:\\
		Sei $\phi$ aus \ComplexityFont{3-SAT} mit gegebener Belegung der Variablen. Dann ist jede Klausel $k_i$ der Form $(x_i \lor y_i \lor z_i)$.\\
		Hieraus konstruieren wir eine neue aussagenlogische Formel $\phi '$ wie oben beschrieben, übernehmen dieselbe Belegung für die übernommenen Variablen und setzen $f$ auf $0$. Anschließend müssen wir noch jedes $c_i$ belegen.\\
		\begin{itemize}
		\item Falls $x_i$ oder $y_i$ wahr ist, so setzen wir $c_i$ auf $0$. Damit ist zum einen die Klausel $k_i'$ wahr, zum anderen gibt es in dieser Klausel mindestens sowohl ein wahres, als auch ein falsches Literal. Desweiteren ist dadurch (unabhängig von $z_i$) auch $k_i''$ wahr, da dort $\bar{c_i}$ enthalten ist. Auch diese Klausel besitzt sowohl ein wahres, als auch ein falsches Literal ($f$).
		
		\item Falls $x_i$ und $y_i$ falsch sind, muss $z_i$ wahr sein. Dann setzen wir $c_i$ auf $1$. Damit ist die erste Klausel ($k_i'$) wahr und besitzt auch wieder mindestens ein wahres, als auch ein falsches Literal. Die Klausel $k_i'$ ist aufgrund von $z_i$ auch wahr, und besitzt auch wieder ein wahres und ein falsches Literal ($f$).
		\end{itemize}
		Also können wir mit der erfüllenden Belegung von $\phi$ auch eine erfüllende Belegung von $\phi'$ konstruieren, sodass $\phi' \in$\ComplexityFont{NOT-ALL-EQUAL-SAT}.\\
		
	\ComplexityFont{NOT-ALL-EQUAL-SAT} $\Rightarrow$ \ComplexityFont{3-SAT}\\
		Sei $\phi'$ aus \ComplexityFont{NOT-ALL-EQUAL-SAT} mit der obig beschriebenen Form. $\phi'$ hat also eine erfüllende Belegung, wobei jede Klausel sowohl ein wahres, als auch ein falsches Literal besitzt. Hier kann es jedoch vorkommen, dass $f$ $0$ ist, aber auch $1$. Wir finden nun eine Belegung für $\phi$:
		\begin{itemize}
		\item	Fall $y=0$:\\
				Dann muss $z_i$ $1$ sein, oder $c_i$ $0$, damit $k_i''$ wahr ist. Ist $z_i$ $1$, dann ist offensichtlich $k_i$ auch wahr.\\
				Ist $z_i$ $0$, so muss $c_i$ $0$ sein. Daraus folgt, dass $x_i$ oder $y_i$ wahr ist (da es sonst keine erfüllende Belegung für $\phi'$ gäbe). Also ist offensichtlich $k_i$ auch wahr.
		
		\item	Fall $y=1$:\\
				Dann muss $z_i$ $0$ sein, oder $c_i$ $1$, damit diese Klausel zwei Literale mit unterschiedlichen Wahrheitswerten besitzt. Jetzt negieren wir die Belegung aller Variablen $x, y, z$ für alle Klauseln.\\
				War $z_i$ ursprünglich $0$, dann ist $z_i$ nach der Negation $1$, weshalb $k_i$ offensichtlich wahr ist.\\
				War $z_i$ ursprünglich $1$, dann ist $c_i$ $1$. Dann muss $x_i$ oder $y_i$ $0$ sein, da sonst die Klausel $k_i'$ keine Literale mit unterschiedlichen Wahrheitswerten hätte. Nach der Negation der Belegung ist also $k_i$ wahr.
		\end{itemize}
		Also können wir für $\phi' \in$\ComplexityFont{NOT-ALL-EQUAL-SAT} zeigen, dass $\phi \in$\ComplexityFont{3-SAT} gilt.
		
\newpage
	
	\section*{Aufgabe 6}
	\begin{enumerate}[label=(\alph*)]
	\item
		Hierzu bauen wir uns einen Verifizierer, welches als Zertifikat einen Vektor $x \in \{0,1\}^n$ übergeben bekommt, welches einfach codiert werden kann als $x_1 ... x_n$.\\
		Unser Verifizierer überprüft nun Folgende Dinge:
		\begin{itemize}
		\item	Eingabe korrekt formatiert (Linearer Zeitaufwand in $n$).
		\item	Überstimmt die Vektordimension $n$ von $x$ überein mit der gegebenen Matrix $A$ und dem Vektor $b$? (höchstens Quadratischer Zeitaufwand in $n \cdot m$, je nach Codierung der Matrix $A$)
		\item	Berechnung von $Ax$ (Quadratischer Zeitaufwand in $n \cdot m$)
		\item	Abgleich, ob $Ax \geq b$ (Linearer Zeitaufwand in $m$).
		\end{itemize}
		Damit läuft der Verifizierer in polynomieller Zeit und \ComplexityFont{$\{-1,0,1\}$-RESTRICTED INTEGER PROGRAMMING} ist in \ComplexityFont{NP}.
		
	\item 
	\end{enumerate}
	
\end{document}

